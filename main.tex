\documentclass[10pt,journal,compsoc]{IEEEtran}

\pdfminorversion=7

\usepackage{amsmath,amsfonts}
\usepackage{algorithmic}
\usepackage{array}
\usepackage[caption=false,font=normalsize,labelfont=sf,textfont=sf]{subfig}
\usepackage{textcomp}
\usepackage{stfloats}
\usepackage{url}
\usepackage{verbatim}
\usepackage{graphicx}


\usepackage{enumerate}
\usepackage{listings}
\usepackage{hyperref}
\usepackage{xspace}
\usepackage{balance}
\usepackage[most]{tcolorbox}
\usepackage{awesomebox}
\usepackage{multirow}
\usepackage{blindtext}



\hyphenation{op-tical net-works semi-conduc-tor IEEE-Xplore}
\def\BibTeX{{\rm B\kern-.05em{\sc i\kern-.025em b}\kern-.08em
    T\kern-.1667em\lower.7ex\hbox{E}\kern-.15emX}}

%.1667


% Custom colors
\definecolor{keywords}{rgb}{0.5,0,0.35}
\definecolor{comments}{RGB}{0,0,113}
\definecolor{red}{RGB}{160,0,0}
\definecolor{green}{RGB}{0,150,0}
 
\lstset{language=Python, 
        basicstyle=\ttfamily\small, 
        keywordstyle=\color{keywords}\bfseries,
        commentstyle=\color{comments},
        stringstyle=\color{red},
        showstringspaces=false,
        %identifierstyle=\color{red},
       }
       


% Pictures
\usepackage{graphicx}       



\newtcbtheorem{obs}{Finding}{%
        theorem name,%
        colback=gray!5,%
        colframe=gray!35!black,%
        fonttitle=\bfseries,title after break={Lemma  -- \raggedleft Continued}%
    }{lem}

\newcommand{\tb}[2]{\tipbox{{\bf Finding #1}. #2}}

\newcommand{\droidxp}{DroidXP\xspace}
\newcommand{\review}[1]{\textcolor{blue}{#1}}
\newcommand{\alert}[1]{\textcolor{red}{#1}}
\newcommand\kn[1]{\textcolor{red}{KN: #1}}
\newcommand\fh[1]{\textcolor{green}{FH: #1}}
\newcommand\rb[1]{(\textcolor{blue}{RB: #1})}

\newcommand{\highlight}[1]{{\color{red}}#1}

\newcommand\raw[1]{\textcolor{red}{#1}\xspace}

\newcommand{\mas}{MAS approach\xspace}

\newcommand{\fm}[1]{\emph{#1}\xspace}

\newcommand{\gps}{\fm{gappusin}\xspace}  % the gappusin family.

\newcommand{\sscore}{Similarity Score\xspace}

\newcommand{\rqa}{What is the impact of considering a larger and diverse dataset on the accuracy of the \mas for malware classification?}

\newcommand{\rqb}{How much gain we obtain on the performance of the \mas for malware classification when considering trace analysis?}

\newcommand{\rqc}{What is the influence of the similarity between the original and repackaged
                  versions of the apps on the performance of the \mas for malware classification?}

\newcommand{\rqd}{What is the influence of the malware family (e.g., \fm{gappusin}, \fm{kuguo}, \fm{dowgin}) on the performance of the \mas for malware classification?}

\newcommand{\rqe}{How much gain we obtain on the performance of the \mas for malware classification when considering its extensions?}


\newcommand{\repack}{RePack\xspace}
\newcommand{\amc}{AndroMalPack\xspace}

\newcommand{\appsSmall}{102\xspace}
\newcommand{\apps}{\textcolor{blue}{4,076}\xspace}
\newcommand{\napps}{\textcolor{blue}{726}\xspace}

\newcommand{\sds}{\texttt{SmallDS}\xspace}
\newcommand{\cds}{\texttt{LargeDS}\xspace}
\newcommand{\nds}{\texttt{DS3}\xspace}
\newcommand{\avt}{\texttt{avclass2} tool\xspace}
\newcommand{\vt}{\texttt{VirusTotal}\xspace}
\newcommand{\se}{security engine\xspace}
\newcommand{\ses}{security engines\xspace}


\newcommand{\fone}{F1-score\xspace}
\newcommand{\fscoreSmall}{0.89\xspace}
\newcommand{\fscoreNew}{0.85\xspace}

\newcommand{\nfscoreSmall}{0.85\xspace}
\newcommand{\nfscoreSmallC}{0.87\xspace}

\newcommand{\fscore}{\textcolor{blue}{0.59}\xspace}
\newcommand{\fscoreC}{\textcolor{blue}{0.49}\xspace}

\newcommand{\malwares}{\textcolor{blue}{2,729}\xspace}
\newcommand{\malwaresP}{\textcolor{blue}{66.95}}
\newcommand{\malwaresN}{\textcolor{blue}{87.98}}
\newcommand{\appsGps}{\textcolor{blue}{1,337}\xspace}
\newcommand{\appsGpsFN}{\textcolor{blue}{1,025}\xspace}



\begin{document}

\title{The Achilles' Heel of the Android Mining Sandbox Approach for Malware Identification}


\author{Francisco Costa, %~\IEEEmembership{Member,~IEEE,}
        Ismael Medeiros, %~\IEEEmembership{Fellow,~OSA,}
        Leandro Oliveira, %~\IEEEmembership{Fellow,~OSA,}
        Jo\~{a}o Cal\'{a}ssio, %~\IEEEmembership{Fellow,~OSA,}
        Rodrigo~Bonif\'{a}cio,
        Krishna Narasimhan,
        Mira Mezini,
        and M\'{a}rcio Ribeiro% <-this % stops a space
\IEEEcompsocitemizethanks{
\IEEEcompsocthanksitem F. Costa, I. Medeiros, L. Oliveira, J. Cal\'{a}ssio, and R. Bonif\'{a}cio are with the 
Computer Science Department, University of Bras\'{i}lia, Bras\'{i}lia, Brazil.
E-mail: \{francisco.costa,ismael.medeiros,leandro.oliveira,joao.calassion\}@aluno.unb.br and \{rbonifacio\}@unb.br.
\IEEEcompsocthanksitem K. Narasimhan and M. Mezini are with the Software Technology Group / TU Darmstadt, Darmstadt, Germany.\protect\\
E-mail: \{kri.nara,mezini\}@informatik.tu-darmstadt.de.
\IEEEcompsocthanksitem M\'{a}rcio Ribeiro is with the Computing Institute, Federal University of Alagoas, Macei\'{o}, Brazil\protect\\
E-mail: marcio@ic.ufal.br}% <-this % stops an unwanted space
%\thanks{Manuscript received April 19, 2005; revised August 26, 2015.}
}



% \thanks{}



%\author{IEEE Publication Technology Department

%\author[1]{Francisco Handrick da Costa}
%\author[1]{Ismael Medeiros}
%\author[1]{Thales Menezes}
%\author[1]{Jo\~{a}o Victor da Silva}
%\author[1]{Ingrid Lorraine da Silva}
%\author[1]{Rodrigo Bonif\'{a}cio}
%\author[2]{Krishna Narasimhan}
%\author[3]{M\'{a}rcio Ribeiro}

%\address[1]{Computer Science Department, University of Bras\'{i}lia, Brazil}
%\address[2]{Software Technology Group, TU Darmstadt, Germany}
%\address[3]{Institute of Computing, Federal University of Alagoas, Brazil}

\IEEEtitleabstractindextext{
\begin{abstract}
Android is the most popular operating system for the mobile platform, and smartphones’ ubiquitous nature in our daily lives has made their security an important concern for researchers and practitioners alike. Among other techniques, previous research results have advocated using the Mining Android Sandbox approach (\mas) to identify malicious behavior in repackaged apps, one recurrent method to inject malicious behavior into android apps. Nonetheless, these previous studies have drawn their conclusions using a small and {\color{blue} out of date dataset} of \appsSmall pairs of original and repackaged apps, threatening the findings w.r.t. external validity and opening the question of whether or not the MAS approach scales to larger datasets. To shed light into these issues, in this paper we report the results of an experiment reproducing the state-of-the-art research that empirically evaluated the MAS approach accuracy. Differently of previous research work, our study uses two datasets (\cds and \nds). The first one is an order of magnitude larger than the dataset used in previous research (\sds) (a total of \apps pairs of apps with a much diverse malware families). The second is not so large when compared with \cds, however more updated, covering a total of \napps repacked apps with detection date from 2018 to 2022). To our surprise, our results reveal that the accuracy rate of the \mas for malware identification drops significantly, when compared \sds and \cds, ($F_1$ score drops from \fscoreSmall to \fscore). However the $F_1$ score remains stable when compared with \nds, (from \fscoreSmall to \nfscoreSmall). After an in-depth assessment, we found that the representative number of malware from the \gps family {\color{red}partially} explains the higher number of samples for which the \mas fails to correctly classify as malware at \cds. {\color{blue}We also investigated the accuracy of two extensions for the \mas: Trace analysis and parameter analysis. The studies demonstrate that when we combining both extensions we improve the accuracy in 10\% at \cds (\fscore to \fscoreC) and just 2\% at \nds (\nfscoreSmall to \nfscoreSmallC)}, still far from the accuracy that previous studies reported. Our findings open the discussion on the possible blind spots that plague the \mas and their accuracy issues when using diverse datasets, and reveal the need for complementing the \mas with other techniques that it could effectively detect a broader class of malware.
\end{abstract}
}

%\begin{IEEEkeywords}
%  Android Malware Detection, Dynamic Analysis, Mining Android Sandboxes
%\end{IEEEkeywords}


\markboth{IEEE Transactions on Software Engineering,~Vol.~XX, No.~X, August~XXXX}%
         {Costa \MakeLowercase{\textit{et al.}}: The Achilles' Heel of the Android Mining Sandbox Approach}


\maketitle


\begin{IEEEkeywords}
Android Malware Detection, Dynamic Analysis, Mining Android Sandboxes.
\end{IEEEkeywords}

\section{Introduction}\label{sec:introduction}

Almost two-thirds of the world use mobile technologies, such as smartphones and tablets, acquired a central role in everyday life in the last decade~\cite{Comscore}\cite{DBLP:journals/tse/MartinSJZH17}. In this context, Android Operation System has dominated this market, with around 150 billions of Android application~\footnote{In this paper, we will use the terms Android Applications, Android Apps and Apps interchangeably, to refer to Android software applications} (apps) downloaded by October 2018 \kn{Do we have something more new} from its marketplaces (Google Play)~\cite{Statista}. Due to this growing popularity, we witnessed an unprecedented growth of  Android apps that perform malicious activities (malware).  This inturn has made security issues in Android apps a relevant research topic. Several techniques have emerged to identify malicious behavior and vulnerabilities in Android apps, such as static analysis algorithms to expose private information leaks~\cite{DBLP:conf/pldi/ArztRFBBKTOM14}, or reveal misuse of 
cryptographic primitives.~\cite{DBLP:journals/tse/KrugerSABM21}

Another alternative for protection from Android malicious behavior consists in the use of dynamic analysis to mine Android sandboxes~\cite{DBLP:conf/icse/JamrozikSZ16}. Such approaches use automated testing tools (i.e., dynamic analysis) to explore apps behavior in terms of access to sensitive resources. A particularly popular way of creating malware is to injecstart with a benign version of an app from an official app store such as Google Play and inject it with code performing malicious activities such as broadcasting credit card information to a private server (\kn{Here add a citation that shows prevalence of repackaged malware}). The state of the art in terms of accuracy in detecting such malware is DroidBot~\cite{DBLP:conf/icse/LiYGC17} which has an accuracy rate of X\%. But their evaluation was performed on a set of $102$ app pairs~\footnote{App pairs here refers to pairs of the same app containing a benign and a malicious repackaged version}. Our internal analysis also revealed that the similarity index between the app pairs used in DroidBot's evaluation was $77\%$. Previous studies from David Lo et al.~\cite{DBLP:conf/wcre/BaoLL18} has revealed that malwares tend to be harder to detect if the similarity index is lower. Both the above mentioned points indicate that the relatively high accuracy reported in earlier evaluations of state of the art mining sandbox approaches may not scale to larger datasets with a more diverse and potentially higher similarity index.

In this paper, we address two open questions that arise. First, we seek to understand what the accuracy of the state of the art in mining sandbox approaches, DroidBot is when presented with a larger dataset ($800$ as opposed to the $102$). This new dataset also has a combined average similarity index of $Y\%$ which is lower than the original $77\%$. Secondly, if the accurary rate has indeed significantly dropped during the reproduction of DroidBot with a more real-world representative dataset what approaches can improve upon the accuracy. 

To answer the first question, we present the results of a study, where we observed the reults of DroidBot on a set from $800$ real apps pairs (benign/malicious). 
 For this reproduction study, we take advantage of DroidXP~\cite{DBLP:conf/scam/CostaMCMVBC20}, a tool suite that supported us not only to integrate DroidBot but also in our study setup and data collection.

To answer the second question, we hypothesize that mining sandbox approaches such as DroidBot perform a superficial analysis of differences between app pairs, i.e., only the difference in the set of sensitive APIs called. We assert that there are two major blindspots in the state of the art:
\begin{enumerate}
    \item The differences between two apps that call the same set of sensitive APIs, but differ in their dynamic call traces between the app's entry point and the sensitive APIs.
    \item The differences in the requested permissions between the two variant of apps.
\end{enumerate}

To this end, we used an auxiliary tool from DroidXP project called DroidXPTrace to compare all traces from an entry point to sensitive APIs call and checking if they are divergences. To observe the differences in the manifest files we harness the standard apk analysis tool from Android~\cite{au2011short}. 

In particular, we investigated the following specific questions in our study: (answered in Section~\ref{sec:results}):

\begin{enumerate}[(RQ1)]
 \item How well does the state of the art in mining sandbox approaches, DroidBot perform with a large dataet of app pairs with a diverse similarity index?
 \item Numerically, how relevant is a dynamic call trace analysis to improving malware detection in support of the mine sandbox approach?
 \item Numerically, how relevant is a manifest files analysis to improving malware detection in support of the mine sandbox approach?
\end{enumerate}
 

Our findings indicate that the accuracy of DroidBot significantly drops (to $Z\%$) when reproduced on a larger dataset of $800$ app pairs with a similarity index of $53.3\%$. Our experiments also reveal that the accuracy of sandbox approaches improve when they are made aware of the differences in the dynamic call trace and difference in permissions requested using the manifest file. Specifically, ... \kn{Please add here the results of path analysis and manifest files individually and their intersection}
During our analysis we also observed another interesting insight that out of the 162 sensitive APIs explored, just xx (xx\%) was injected at most repackage apps. It indicates that we have a small set of sensitive APIs used generally by malicious developers, pointing out that researchers can concentrate their effort on this specific set of sensitive APIs to improve the security of Android apps. 
In summary, this paper makes the following contributions:

\begin{enumerate}[1.]
\item A reproduction of the state of the art in mining sandbox approaches, Droidbot scaled in terms of number of app pairs and similarity index.
\item A broad comprehension about the role of trace analysis and static analysis on Android manifest file in improving the accuracy Android sandbox approach.
\item A in depth look into the kind of sensitive APIs that plague most repackaged apps.
\item A reproduction package of the studies online. Scripts for statistic analysis are also available.\footnote{https://github.com/droidxp/paper-droidxp-replication.git}
\end{enumerate}


The rest of the paper is organized as follows.....\kn{Here, please describe each section in one sentence}





\section{Background and Related Work}\label{sec:background}

%% In this section, we introduce the concepts and terminology that are necessary to understand the reminder of this paper. First, Section~\ref{sec:sand} introduces some background information about \emph{sandboxes} within the security context. Section~\ref{sec:repackage} presents background information about repackaged application and how they introduce malicious behavior.
%% Finally, in Section~\ref{sec:android-sandbox} we review the \emph{mining sandbox approach} for detecting repackaged Android apps.

The Android bytecode language~\cite{DBLP:conf/issta/WangGMC15} favors reverse engineer tasks. That is, software developers can easily reverse-engineer real apps (benign), modify their contents by inserting malicious code (malware), repackage them with the malicious payloads, and re-advertise them in the official market, Google Play Store, or other markets. Repackage Android apps can leverage the popularity of real apps, to increase its propagation and spread malware.  
Repackaging has been raised as a great security problem in Android ecosystem by stakeholders in the app development industry and researchers. There are works~\cite{DBLP:conf/sigmetrics/ViennotGN14} claiming that about 25\% of Google Play Store app content is repackaging apps. Nevertheless, all the workload to detect and remove malware from markets by the stores (official and no official), have not been accurate enough to address the problem as fast as possible. As a result, repackaged Android apps follow threatening security and privacy of unsuspicious Android app users, beyond compromising the copyright of the original developer~\cite{DBLP:journals/access/KimLCP19}. Aiming at
mitigating this threat, static and dynamic analysis approaches have been proposed.

\subsection{Dynamic and Static Analysis on Android apps}\label{sec:analysis}

There is a large body of work that explores the use of program analysis techniques to detect malware. 

Several works have been proposed to detect malware based on sensitive method calls and permission control~\cite{DBLP:conf/mobicom/WeiGNF12,DBLP:conf/asiajcis/WuMWLW12,DBLP:conf/sp/LiDLDG21}. Cai et al.~\cite{DBLP:journals/tse/CaiR21} presented a longitudinal study on Android apps focusing on run-time behaviors. However, this work does not focus specifically on malware detection but on general security gaps in apps by considering only benign apps. Fangfang et al.~\cite{DBLP:conf/wisec/ZhangHZW014} proposed ViewDroid, which models the UIs of Android apps as a directed graph. Although ViewDroid also works by comparing app pairs to identify repackaged apps, their focus is UI centric.

On static analysis approaches exploring Android Manifest files, Kim et al. proposed RomaDroid~\cite{DBLP:journals/access/KimLCP19}.  Their approach does not consider the structural context in Manifest files, but rather treat the files as sequence of strings and perform a lowest common subsequence (LCS) based approach to detect repackaged apps. Au el al.~\cite{DBLP:conf/ccs/AuZHL12} also apply static analysis on Android Manifest files to detect vulnerabilities in Android apps. They do this by mapping requested permissions to sensitive API calls in the code.

Li et al.~\cite{DBLP:journals/tifs/0029LBKTLC17} provided a systematic knowledge on Android malware by conducting an empirical study comparing malicious repackage app with their benign counterparts (1,497 app pairs). They found that the majority of Android malware are repackaged versions of benign apps that donot do anything complex modifications, many times simply reusing library code.

In the domain of detecting repackaged apps by comparing app pairs, Crussell et al.~\cite{DBLP:conf/esorics/CrussellGC12} proposed  DNADroid, which compares program dependence graphs, and Zhou et al.~\cite{DBLP:conf/codaspy/ZhouZJN12} DroidMoss which detects and analyzes repackaged apps using a fuzzy hashing technique. 

\subsection{Mining Android Sandboxes}\label{sec:android-sandbox}

A \emph{sandbox}
is a well-known mechanism to secure a system and forbid a software component to access
resources that it is not allowed to. Sandboxes have also been used to build an isolated
environment on an electronic device within which applications cannot affect other programs, the network, or other device data~\cite{DBLP:journals/peerj-cs/MaassSCS16}. In this scenario, the idea of using sandboxes emerged from the
need to testing unsafe software, possible malware, without worrying about the integrity of the
device under test~\cite{DBLP:conf/esorics/BordoniCS17}, shielding the operation system from any security issue.
To ensure safety, a sandbox environment should have the minimum requirements to run the
program (make sure the program will not escape the sandbox), and make sure it will never
assign the program greater privileges than it should have, working with the principle of
\emph{least privilege}. This principle ensures unauthorized access to resources,
improving the system's overall health. Regarding Android ecosystem, the principle
of the \emph{least privilege} is ensured by sandboxing too,
where apps never access direct resources or data of other apps. Access to sensitives resources
like contacts list is granted through specific APIs (Application Programming Interface),
which are managed by permissions~\cite{DBLP:journals/corr/abs-2109-06613}. 

%% The main market source for Android apps is Google Play Store. Unfortunately, it has
%% a flexible policy regarding the process of publishing apps, and therefore, many Android apps are removed from the
%% store because of issues related to malware\cite{DBLP:conf/msr/WangLL0X18}. Google Play tries
%% to minimize unauthorized access to sensitive resources by malicious apps,
%% listing each app with its requested permission. {\color{red}Those permissions are presented to Android
%% users at app installation moment since version 6}. However, some works presented that most users are careless regarding these permissions since they are only interested to run the app~\cite{DBLP:conf/soups/FeltHEHCW12}. This represents a great security breach since malware usually asks for more permissions than their APIs normally would require~\cite{DBLP:conf/ccs/FeltCHSW11}.



The Mining Android Sandbox approach~\cite{DBLP:conf/icse/JamrozikSZ16} aims at automatically
building a sandbox from dynamic analysis (i.e., using automatic test generation tools).
The main idea is to explore apps based on their calls to sensitive APIs.
Thus, sandboxes build upon these calls to create safety rules and then block future
calls to other sensitive resources, which diverge from those found in the first exploratory
phase. Using a test generation tool named Droidmate~\cite{DBLP:conf/icse/JamrozikZ16},
Jamrozik et al.~\cite{DBLP:conf/icse/JamrozikSZ16} proposed the first mainstream
implementation of the Mining Android Sandbox approach, called Boxmate. 
Boxmate records the occurrences of calls to sensitive APIs and the event that triggers these calls,
like a button click. It is possible to configure Boxmate to record events associated with each sensitive call as
tuples (event, API). Jamrozik et al. argue that, in this way, Boxmate generates finer granularity results which
might reduce false alarms, even with the presence of reflection which is quite commonly used in
malicious apps~\cite{DBLP:conf/issta/0029BOK16}. 

Figure~\ref{fig:mineSandbox} extracted from paper~\cite{DBLP:conf/wcre/BaoLL18}, summarises the approach. 

\begin{figure}[ht]
\centering
\includegraphics[scale=0.30]{images/mineSandbox.pdf}
\caption{Mine Sandbox. Extracted from~\cite{DBLP:conf/wcre/BaoLL18}}
 \label{fig:mineSandbox}
\end{figure}

The Mining Android Sandbox approach accuracy suggests a close relationship with the
efficiency of the exploratory phase. The more efficient the test generator
tools (for instance, in terms of code coverage), the more accurate would be the resulting
sandbox. Besides being used to generate Android sandboxes, the Mining Android Sandbox approach is also effective 
to classify malwares~\cite{DBLP:conf/wcre/BaoLL18}.  In this scenario, the \emph{effectiveness} of the approach
is estimated in terms of repackage/malware identification.
Previous studies~\cite{DBLP:conf/wcre/BaoLL18,DBLP:conf/scam/CostaMCMVBC20} that explored the effectiveness of mine sandboxes,
investigating and comparing the performance of different test generator tools,
with diverse exploratory strategies.


The focus of our paper is in approaches that mine android sandboxes to detect Android Malware.
There is a vast body of research in this direction. 
Bao et al.~\cite{DBLP:conf/wcre/BaoLL18} conducted an empirical study to investigate the effectiveness of android sandbox approaches
for detecting malware, exploring test generation tools, including Droidmate. The authors found that in general, the sandboxes constructed by test generator tools can detect more than $70$\% malicious apps in a dataset comprising $102$ pairs (benign/malicious). The study also presented that among 5 test generation tools used, DroidBot~\cite{DBLP:conf/icse/LiYGC17} leads to the most efficient sandbox.
Le et al.~\cite{le2018towards} extend the work of Bao et al. by combining more categories of APIs, and also considering the impact of
actual arguments on the sandboxe definition.
None of the aforementioned studies ~\cite{DBLP:conf/icse/JamrozikSZ16,DBLP:conf/wcre/BaoLL18,le2018towards} neither characterize the APIs included on the repackage versions nor investigate the possibility that trace analysis using call graph or analysis of the manifest file could complement the mining sandbox approach for malware identification.

Hence, our work, although closely related to previous studies, differs from them in several aspects. We present a broader assessment of the effectiveness of the mining sandbox approach for malware identification, resulting in the first in depth characterization of the calls to sensitive APIs that frequently appear in the repackage version of the apps. Our assessment is also more comprehensinve: instead of considering $102$ pairs of benign/malign apps, we execute our study consider {\color{red}800} pairs of apps. We also explore two possible strategis to complement the mining sandbox approach for malware identification. The first explores the impact of dynamic call graphs through the comparisons of traces from entry point to the calls to sensitive APIs that result from the executions of the benign and malign versions of an app. The second explores suspicious change patterns in the Android Manifest file of the apps, using a simple, yet effective, static analysis approach.

\section{Experimental Setup}\label{sec:experimentalSetup}

%\todo[inline]{Review the following paragraph later.}

In this section, we describe our study settings. First we present how we mined the data-set of Android app pairs that will serve as a benchmark for our study (Section~\ref{sec:dataset}).  Then, we describe the data collection procedures used to perform the study (Section~\ref{sec:dataCollectionProc}). Section~\ref{sec:dataAnalysisProc} presents our data analysis procedure describing the adaptations we performed on the base setup in order to facilitate the individual analyses (Section~\ref{sec:malwaresetup}, \ref{sec:pathsetup}, \ref{sec:manifestAnalysis}). We finally discuss the hardware setup and configurations used at our experiment at (Section~\ref{sec:hardware}).


\subsection{Malware Dataset}\label{sec:dataset}

The majority of existing Android malware correspond to repackaged versions of benign apps from official stores\cite{DBLP:conf/codaspy/ZhouZJN12} 
%\rb{(we should include a citation here)}
, though with code that performs malicious tasks. Zhou et al.~\cite{DBLP:conf/sp/ZhouJ12} curated a wide-ranging collection of malware (named Androzoo)~\cite{DBLP:conf/msr/AllixBKT16}, where 80\% of the samples are repackaged versions of benign apps. Our research benefit from Androzoo, since it contains both versions of the apps: the benign and the repackaged version of the apps. To evaluate mining sandbox approaches for malware detection, we need these pairs of apps (benign and malign versions). During the exploratory phase, we collect relevant information (such as the set of calls to sensitive APIs) while executing a test case generation tool targeting the benign version of the app. Later, during an analysis phase, we execute the test case generation tool again, though targeting the malicious version of the app.

\todo[inline]{We should address these numbers in the threats to validity section. We dropped from a original dataset with 1831 pairs to 824 pairs only, due to problems that happened during the instrumentation / execution phases of the apps. Is it possible to fix this problem in the future?}

Besides having this necessary pairs of apps, Androzoo is a large collection of Android apps mined from multiple markets, including the official Google Play Store~\footnote{\url{https://play.google.com/store}}. We limited our download of app pairs to $20$ hours, resulting in $1831$ unique pairs. However, we could only instrument $1395$ apps from the original set using DroidFax, and execute only $800$ of them in the Android emulator we use in our experiments (Pixel 4, API 28). That is, due to errors that occurred either during DroidFax instrumentation or during the execution of the apps in the exploratory or analysis phases, we removed from the original dataset $1031$ apps, leading to a final dataset that contains $800$ pairs of Android apps. Figure~\ref{fig:stores} presents the markets where most of malicious version apps were collected (top 10) according to Androzoo, including several from the official app store of Android, Google play store ($169$).

%\todo[inline]{We must update Figure~\ref{fig:stores}. I would also consider removing it from the paper,
%  and just present some numbers directly in the text. Something like. ABC\% of the apps
%  come from Google Play Store, \ldots}


\textbf{Differences from previous work: } The previous state of the art work in mining sandbox approaches~\cite{DBLP:conf/wcre/BaoLL18} has a dataset with 102 samples, an average similarity index of $77\%$, and $5$ different malware categories. In contrast, our dataset is larger and more representative. It has 800 samples, with more than $10$ malware types, and an average similarity index of $62.47\%$. To calculate similarity index~\cite{DBLP:conf/wcre/0029BKT16} we used SimiDroid~\cite{DBLP:conf/trustcom/0029BK17}. It quantifies and qualifies the similarity based on (a) the methods that are either identical or similar in both versions of the apps, (b) methods that only appear in the malicious version of the apps (new methods), and (c) methods that only appear in the benign version of the apps (deleted methods).

%% \begin{itemize}
%%     \item \textit{identical}: a given method is considered identical if both versions have the same signature and the same implementation
%%     \item \textit{similar}: a given method is similar if both versions have the same signature but not the same implementation.
%%     \item \textit{new}: a given method is new if it is present only in the malicious version of the app.
%%     \item \textit{deleted}: a given method is deleted if it is present only in the benign version of the app.
%% \end{itemize}


%\kn{Here cite the previous work and describe the dataset there like size, similarity index etc.}... In contract our dataset is larger and more representative... \kn{Here describe what we mean my similarity index after removing 3.4 and also describe that we are different in size, similarity index and malware types covered}.


\begin{figure}[ht]
\centering
\includegraphics[scale=0.4]{images/stores.pdf}
\caption{Markets where malware was discovered.}
 \label{fig:stores}
\end{figure}

%\todo[inline]{Since the choice for DroidBot is not related to the goal of this section (Constructing the Dataset),
%  I would rather move this whole paragraph for a different place in the paper.}

%\kn{I have removed the following as it is inconsistent with our current story. We no longer perform just a study, we offer an approach. And we no longer study all four, only the top one DroidBot and built on top of it. }

%To perform an in-depth analysis of how various factors influence the quality of android sandboxes, we could have used several state of the art test generation tools like: Monkey~\cite{Monkey}, Droidmate2~\cite{DBLP:conf/kbse/BorgesHZ18}, Droidbot~\cite{DBLP:conf/icse/LiYGC17} and Humanoid~\cite{DBLP:conf/kbse/LiY0C19}. However, we considered to use just Droidbot, since several study show that it achieved better performance results when mining sensitives resources in comparison with the state of the art~\cite{DBLP:conf/wcre/BaoLL18,DBLP:journals/jss/CostaMMSSBNR22}. Its applies a depth-first strategy (DFS)~\cite{DBLP:conf/oopsla/AzimN13} to dynamically build GUI models, collecting GUI information and running process information.


%\todo[inline]{Should we call the following section Data Collection? Yes, I will change it.}

\subsection{Data Collection Procedures} \label{sec:dataCollectionProc}


\begin{comment}

\subsection{Infrastructure}\label{sec:infra}

%\kn{We currently only use one test generation tool. The following paragraph needs to be adjusted to reflect that} 
%For our study, we require an extensive infrastructure that will allow us to integrate different test case generation tools. To this end, we used and extended% DroidXP\cite{DBLP:conf/scam/CostaMCMVBC20}, a benchmark tool that allows researchers and developers, to integrate and compare test case generation tools for% mining sandboxes.

Using DroidXP, test case generation tools can be integrated and deployed easily~\footnote{The DroidXP benchmark is available at https://github.com/droidxp/benchmark}. Another reason to use DroidXP is that it relies on DROIDFAX~\cite{cai2016understanding}, which allowed us to instruments Android apps and collect important information about their execution. For example, this can be used to query for the calls to the sensitive APIs from an app during executions which is important to monitor malicious behavior. 

%\kn{I have commented out the following and shortened it as this is too verbose information about DroidXP's tooling}
The DroidXP relies on a Simple Command Line (CLI) that  provides commands for execution of different experiments with several setup, following parameters:
\newline
\newline
\textbf{DroidXP:} python main.py [-h] [--list-tools] [-tools] toolName [-t] timeSeconds [-r] repetition [--list-outputs] [--output] output [--debug] [--version]\newline
\newline
One can show help message [-h], list available test generation tools [--list-tools], test tools used in the experiment (toolName), threshold of the execution time in the experiment (timeSeconds), number of repetitions used in the experiment (repetition), list available output formats [--list-outputs], output formtat that will be used to show results (default: basic) (output),
run in DEBUG mode (default: false) [--debug] and print DroidXP version [--version].

Follow an example of a command line that executes DroidXP for three minutes (180 seconds), at test generation tools Monkey, repeating all the performance three times, and outputting the final result at a CSV file format:
\newline
\newline
\textbf{python main.py -tools monkey -t 180 -r 3 --output csv}
\newline
\end{comment}


We take advantage of the DroidXP infrastructure~\cite{DBLP:conf/scam/CostaMCMVBC20}
for data collection. DroidXP allows researchers to compare 
test case generation tools for malware identification using the
mining Android sandbox approach. Although the comparison of test
case generation tools is not the goal of this paper, DroidXP
is also useful for supporting the following steps of our study.

\begin{description}
 \item[(Step 1) Instrumentation:] In the first step,
we configure DroidXP to instrument all pairs of apps in our dataset.
Here, we instrument both versions of the apps (APK files) so that
we can collect relevant information during their execution. Under the hood, DroidXP leverages
DroidFax~\cite{DBLP:conf/icsm/CaiR17a} to instrument the apps and collect static
information about them. To improve the performance across multiple executions,
this phase executes only once for each version of the apps in our dataset.

\item[(Step 2) Execution:] In this step, DroidXP first installs the (instrumented) version of the APK files in the
  Android emulator we use in our experiment and then starts a test case generation tools for executing the app. To perform the apps executions, our study builds upon DroidBot~\cite{DBLP:conf/icse/LiYGC17} which is considered the state of the art in mining sandbox approaches in terms of accuracy. To also ensure
  that each execution gets the benefit of running on a fresh Android instance without biases that could stem out of history,
  DroidXP wipes out all data stored on the emulator that has been collected from previous executions.

%\kn{Is the result analysis into logcat done by DroidXP or an extension by us? In general, it is unclear what is existing in DroidXP already and what we do. Isnt logcat a separate tool. If this is done by DroidXP, please make it clear that DroidXP invokes logcat on its own}%
\item[(Step 3) Data Collection:] During the execution, DroidXP collects all relevant information (such as calls to sensitive APIs,test coverage metrics, and so on). We use this information to understand the performance of the mining sandbox approach for malware detection. We extended DroidXP with a new component (DroidXPTrace), which allows us to investigate dynamic call graphs from the outcomes of a DroidXP execution. This was not possible using vanilla version of DroidXP %\rb{(I am note sure about this sentence)}.
%  \fh{I think that we have to separate this data collection section. One for sensitive APIs extract and other for trace extract. The 3 steps presented here must talk about just the first extract, and we must create another one for the second.  }
\end{description}

%\todo[inline]{I am not really happy with the previous structure of this section. Trying to refactor it in two sections: Data Collection Procedures and Data Analysis Procedures.}

\subsection{Data Analysis Procedures} \label{sec:dataAnalysisProc}

\subsubsection{Malware Identification} \label{sec:malwaresetup}
Figure~\ref{fig:sensitiveAPI} illustrates the mining sandbox approach for
malware identification. We leverage DroidXP to collect the set of sensitive APIs the versions (benign/malign) of the apps call. As a first step, given one benign app version,
our approach first collects all calls to sensitive APIs from the app code through static analysis. Then, during the execution step,
we use dynamic analysis to collect all calls to sensitive APIs during the DroidBot test case execution. We configure DroidXP to execute DroidBot for a
period of $3$ minutes. This second analysis is also important since it can disclose some sensitive APIs calls ignored by static analysis.
Malware may use dynamic features (such as reflection) to introduce malicious behavior, which can change the behavior of of the apps at runtime~\cite{DBLP:journals/spe/ZhangLTX18,DBLP:journals/tosem/LiTX19}.
The instrumentation step is made once, while we execute the dynamic analysis $3$ times. The result of instrumentation and all executions is finally joined,
forming a final set of that contains all identified calls to sensitive APIs coming from the benign version of the app. We carry out the same procedure for the malicious version of the apps,
creating a distinct set of calls to sensitive APIs (now coming from the malicious version of the apps). In
a final step, we compare the two sets of calls to sensitive APIs. We use the following rules to
check for a malicious behavior. 

\begin{enumerate}
    \item If the difference between the two final sets is an empty set, we cannot distinguish the benign and malicious version of the app (false negative).
    \item If the difference between the two final sets contains at least 1 element (injected in the malicious version), we successfully distinguish the benign and the malicious version of the app (true positive).
\end{enumerate}

In addition, we also benefit from this procedure to identify the calls to sensitive APIs that are more frequently injected by the malicious version of the apps
in our dataset. 

%\fh{here I have to insert the figure}

\begin{figure}[ht]
\centering
\includegraphics[scale=0.4]{images/sensitiveAPIdiff.pdf}
\caption{All procedure for malware identification using sensitive API set diff.}
 \label{fig:sensitiveAPI}
\end{figure}



%\todo[inline]{Handrick, could you please write about the malware identification procedure?}

\subsubsection{Trace Analysis} \label{sec:pathsetup}

%\todo[inline]{Handrick, is it possible to run the Trace Analysis for all apps?}

As we discussed in the previous section, we take advantage of DroidXPTrace to build the dynamic call graphs that characterize the execution of each version of the apps in our dataset. Our goal
here is to explore how many pairs of apps call the same set of sensitive APIs, though using different call
traces. We conjecture that differences in the traces might be used to complement the mining sandbox
approach for malware identification. As such, here we execute the trace
analysis for all app pairs of our dataset, and check if there is the situations in which the basic version of the mining sandbox approach was not able to correctly classify the malign version of an app as a malware, however it has a different execution trace. For detecting different trace, we performed an evaluation of the dynamic call graph of each pair. Our procedure checks if there is some new node, representing a new sensitive API at malicious version, or a new edge($x$, $y$), where $x$ and $y$ indicates a method $x$ calling a sensitive method $y$.

%\rb{(Not sure if this is the right decision. I do not see any problem in running
%  this study for all pairs.)}. 
%% we  investigate those app pairs that were not described as a malware during the exploratory step, i.e,
%% the test generation tool DroidBot collected the same set of sensitive APIs for both version. If a dynamic call graph
%% of these app pairs presented different traces from entry point to sensitive APIs call at both versions, we suspect this to indicate presence of malware.
Figure~\ref{fig:callGraph} illustrates an example of benign and malicious call graphs.
Although both app versions access the same set of sensitive resources, the
malicious version follows a different execution trace. 


\begin{figure}[ht]
\centering
\includegraphics[scale=0.25]{images/maliciousCallGraph.pdf}
\caption{Illustrative example of the trace analysis. In this case, both versions call the same set of sensitive APIs. Nonetheless,
 the traces between the entry point and the calls to sensitive APIs diverge.}
 \label{fig:callGraph}
\end{figure}


\subsubsection{Manifest file analysis}\label{sec:manifestAnalysis}

Our work also observed the Manifest file of all app pairs. These files present essential security information which the Android system must make available before executing an app. Among other things, an Android app must specify in the Manifest file the list of necessary permissions and components . Unfortunately, Manifest files are not considered in the basic mining sandbox approach for malware detection, in spite of the fact that one can easily modify such a file and then repackage it in a malicious version of an app~\cite{DBLP:journals/corr/abs-1208-4536}. For instance, by inserting new permission requests or component capabilities (actions). Such injections can happen either manually or through automated scripts.

%\todo[inline]{Looks like a naive approach. Note that, malware developers often use advanced techniques in order to hinder security experts detecting the presence of a malicious change in an app---using both static or dynamic analysis. The approach detailed here might be efficient only for poor designed malware. I am discussing this because we are seriously considering submitting this paper to a security venue. The people involved in Usenix Security are dealing with more advanced situations, like malware that changes its behavior when it detects that it is running in um emulator, for instance.}

Automated process sometimes generate Manifest file with duplicate permission requests, as the original app might already contain the permission request. Such duplication also happens when repackaged apps add a component with a capability which was requested by another component. On top of this, our manifest analysis also allows us to monitor suspicious new permissions or excessive amounts of permissions. 

To extract and analyze the Manifest file, we used an in-house Android SDK analyzer called \textit{apkanalyzer}\footnote{apkanalyzer is included in the Android SDK Tools package and provides insight into the Android app components.}. We implemented a python script that uses \textit{apkanalyzer} and computes which malicious apps had duplicated request permission or duplicated actions. The script also extracts how many permissions were requested by all apps, to help us monitor potential suspicious features. Listing~\ref{lst:androidManifestDupli} and Listing~\ref{lst:androidManifestAction} present an example of duplicated permission extracted from malicious version of the app \textbf{[com.ifeel.frogjump]}, and an example of duplicated component capabilities from malicious version of the app \textbf{[com.koushikdutta.superuser]} respectively:

%\kn{app-8 according to what table? reference please}. 

\begin{lstlisting}[caption={Example of duplicated permission from malicious version of app (com.ifeel.frogjump)}, language=Java,
    basicstyle=\fontsize{6}{5}\selectfont\ttfamily,
    label={lst:androidManifestDupli}]

13:M >  <uses-permission
        android:name="android.permission.READ_PHONE_STATE" />

16:M >  <uses-permission
        android:name="android.permission.ACCESS_COARSE_LOCATION" />

19:M >  <uses-permission
        android:name="android.permission.INTERNET" />

22:M >  <uses-permission
        android:name="android.permission.ACCESS_NETWORK_STATE" />
    .
    .
    .
134:M > <uses-permission
        android:name="android.permission.INTERNET" />

137:M > <uses-permission
        android:name="android.permission.WAKE_LOCK" />

140:M > <uses-permission
        android:name="android.permission.READ_PHONE_STATE" />

143:M > <uses-permission
        android:name="android.permission.ACCESS_NETWORK_STATE" />

146:M > <uses-permission
        android:name="android.permission.WRITE_EXTERNAL_STORAGE" />

149:M > <uses-permission
        android:name="android.permission.ACCESS_WIFI_STATE" />
\end{lstlisting}

\begin{lstlisting}[caption={An example of duplicated component capability from malicious version of app (com.koushikdutta.superuser)}, language=Java,
    basicstyle=\fontsize{6}{5}\selectfont\ttfamily,
    label={lst:androidManifestAction}]

101:M > <receiver
102:M >  android:name=".SuCheckerReceiver">
103:M >  <intent-filter>
104:M >    <action
105:M >      android:name="android.intent.action.BOOT_COMPLETED" />                 
106:M >  </intent-filter>
107:M > </receiver>
108:M > <receiver
109:M >  android:name=".PackageChangeReceiver">
110:M >  <intent-filter>
111:M >    <action
112:M >      android:name="android.intent.action.BOOT_COMPLETED" />
113:M >    <data
114:M >      android:scheme="package" />
115:M > </intent-filter>
     .
     .
     .
\end{lstlisting}

Listing~\ref{lst:androidManifestDupli} suggests that the four first requested permission could have been added automatically. Listing~\ref{lst:androidManifestAction} presents a duplicated component that might also suggest that the first was newly injected by a naive hacking script.


%% We extended DroidXP with a new component (DroidXPTrace) that allows us to build a dynamic call graph from the outcomes of a DroidXP execution. This was not possible using vanilla version of DroidXP \rb{(I am note sure about this sentence)}. DroidXPTrace, using the result from Phase 3 of DroidXP, creates a call graph of app pair, exploring traces from its entry point to sensitive API access. In the end, it compares both call graphs (benign and malicious app version), and generated a JSON file recording the following information:
%\kn{I have rephrased the previous paragraph, please check if it is still accurate}. 

%\kn{Overall the following bits in the rest of this subsection are a bit confusing.  You use the word trace and graph interchangeably. is it a graph or a trace. And finally the connection to paths taken comes out of nowhere. I propose rephrasing this part with an example data from some app. Only one of the columns is enough. Even if the example is an artificial one it is fine. Important is that it becomes clear what data is recorded. }
%% \begin{itemize}
%% \item \texttt{benign}: Name of log file from the benign app version
%% \item \texttt{malign}: Name of log file from the malicious app version
%% \item \texttt{benignGraphs}: call graphs that contain traces with access to sensitive methods in the benign version of the app.
%% \item \texttt{malignGraphs}: call graphs that contain traces with access to sensitive methods in the malicious version of the app.
%% \item \texttt{methodsAccessedOnlyByMalign}: The sensitive methods that are accessed only by the malicious version of the app. This information is important to identify if a particular sensitive call that is undetected by a sandbox occurs only in a malicious version.
%% \item \texttt{benignGraphContainsMalignGraph}: Comparison between benign and malicious sub-callgraphs
%% \item \texttt{hasDifferentTraces}: whether the benign and malicious app version have different traces to sensitive resources. 
%% \end{itemize}

%%  This information helped us to explore traces between an app's entry point and its access to sensitive methods during run time.
%%  \kn{What I have commented below is talking about the study itself and not the setup. Please ensure other such instances are removed if I have missed them.}
%%  %This investigation shows that in several situations, although both app versions (benign-malicious) access the same set of sensitive resources, they access them using different traces, which could mean malicious sensitive resource access.





\subsection{Execution Settings}\label{sec:hardware}

We deployed our experiment on a 32-Core, AMD EPYC 7542 CPU, 512 GB RAM, storage Samsung SSD 970 EVO 1TB machine running a 64-bit Debian  GNU/Linux 11. We also configured our emulator to run all selected apps on Google Android version 9.0, API 28, 512M SD Card, 7GB internal storage, with X86 ABI image.

For our study, we configured DroidXP to run each of the $800$ app pairs using Droidbot for three minutes. To mitigate noise, we repeated the full process three times which took in total ($800 \times 2$ apps (benign/malicious) $\times 3$ runs $\times 3$ min) + ($800 \times 2$ apps $\times 1.5$ min for emulator reboot) $\apeq 240$ machine hours.
%\kn{This number needs to be changed here too. I suggest introducing a macro for this and using it everywhere to make sure everytime our experiment evolves, we don't need to change this everywhere} 

%\kn{Initial? Was there a follow up study, if not please use only study. This is feedback for all instances with the word initial}

Although it was possible to run more than 10 emulators in parallel on one physical machine, to avoid any interference resulting from context switching within the operating system, we chose to run one emulator at a time. Hence, all evaluation processes took around $12$ days and additional $5$ days for environment configuration.

% In the following subsections, we describe the setup required to perform custom analysis that were central to our study.

%% \subsection{Sensitive API setup} \label{sec:sensitivapi}
%% \fh{Here the text talk about false positive. We have to change to talk about how we extract the most sensitive APIs used by malicous apps }
%% \kn{I am ignoring this subsection for review as it is no longer relevant. Please add some details about the sensitive APIs collection so it can be extended upon.}
%% To perform the false positive analysis we used the infrastructure of DroidXP, described in section~\ref{sec:infra}. After finishing \textbf{Phase 2 (execution)}, we have all explorer information about the apps under analysis by Droidbot test generation tools.

%% Our analysis concert just at the benign version of apps, since false positive alarm could occur when some expected behavior from the benign app version is not seen during mining step. Hence, as a first step we collect all sensitive methods (SM) by the union of three execution of mine sandbox at all benign apps version. With this step, we could have a base set of sensitive methods called by each benign version from all 95 apps explored in our study. After, we executed mine sandbox again at the same set of benign apps version seven times. At each execution we observed the set of sensitive methods called, and compare them with the initial set of sensitive methods composed by the union of methods explored by the first three executions. In the end, we executed 7 tests, checking if the difference between all the these sets are empty sets, as below:\newline

%% \begin{enumerate}[(Test 1)]
%%  \item {$\left\{\left\{SM01\right\} \cup \left\{SM02\right\} \cup \left\{SM03\right\}\right\} \textit{diff} \left\{SM04\right\} = \emptyset$}
%%  \item {$\left\{\left\{SM01\right\} \cup \left\{SM02\right\} \cup \left\{SM03\right\}\right\} \textit{diff} \left\{SM05\right\} = \emptyset$}
%%  \item {$\left\{\left\{SM01\right\} \cup \left\{SM02\right\} \cup \left\{SM03\right\}\right\} \textit{diff} \left\{SM06\right\} = \emptyset$}
%%  \item {$\left\{\left\{SM01\right\} \cup \left\{SM02\right\} \cup \left\{SM03\right\}\right\} \textit{diff} \left\{SM07\right\} = \emptyset$}
%%  \item {$\left\{\left\{SM01\right\} \cup \left\{SM02\right\} \cup \left\{SM03\right\}\right\} \textit{diff} \left\{SM08\right\} = \emptyset$}
%%  \item {$\left\{\left\{SM01\right\} \cup \left\{SM02\right\} \cup \left\{SM03\right\}\right\} \textit{diff} \left\{SM09\right\} = \emptyset$}
%%  \item {$\left\{\left\{SM01\right\} \cup \left\{SM02\right\} \cup \left\{SM03\right\}\right\} \textit{diff} \left\{SM10\right\} = \emptyset$}
 
%% \end{enumerate}

%% We observed if all execution explored the same sensitive methods at each benign app in our dataset (95 app pairs). We consider that occur a false positive alert when at least one of seven tests fail, i.e. if one of the 7 tests returns a non-empty set.

%% As a second step, we also choose a random execution, and check if the set of sensitive methods collected by this execution, is the same of other 9 sensitive methods set, collected by others 9 executions. At this second step, the fifth execution was the choose one as a base execution for tests, as described below:\newline

%% \begin{enumerate}[(Test 1)] \setcounter{enumi}{7}
%%  \item {$\left\{SM05\right\} \textit{diff} \left\{SM01\right\} = \emptyset$}
%%  \item {$\left\{SM05\right\} \textit{diff} \left\{SM02\right\} = \emptyset$}
%%  \item {$\left\{SM05\right\} \textit{diff} \left\{SM03\right\} = \emptyset$}
%%  \item {$\left\{SM05\right\} \textit{diff} \left\{SM04\right\} = \emptyset$}
%%  \item {$\left\{SM05\right\} \textit{diff} \left\{SM06\right\} = \emptyset$}
%%  \item {$\left\{SM05\right\} \textit{diff} \left\{SM07\right\} = \emptyset$}
%%  \item {$\left\{SM05\right\} \textit{diff} \left\{SM08\right\} = \emptyset$}
%%  \item {$\left\{SM05\right\} \textit{diff} \left\{SM09\right\} = \emptyset$}
%%  \item {$\left\{SM05\right\} \textit{diff} \left\{SM09\right\} = \emptyset$}
%% \end{enumerate}

%% As the first analysis, we also consider a false negative occurrence, if one of the 9 remaining tests returns a non-empty set.


\section{Results and discussion}\label{sec:results}

In this section, we detail the findings of our study. We present the results of the first study in Section~\ref{sec:testGeneration}. It could estimate the performance of the test generation tools for mining Android  Sandboxes, in terms of malware detection. Section \ref{sec:path} present the result of Paths analysis, and Section~\ref{sec:manifest} Manifest file analysis. Section~\ref{sec:implications} cross some results, and present some implications.

\subsection{Effectiveness of Test generation tools on Detecting Malicious apps}\label{sec:testGeneration}

Given a pair of apps, we first perform an exploratory phase at both app versions at a specific test generation tool, for three minutes, as detailed before. After execution, DroidXP produces a dataset with the sensitive APIs that both app versions call. We consider that a test generation tool could construct a sandbox able to detect or do not a specific malware, by comparing the calls to sensitive APIs made by the benign and malign versions of the apps. DroidXP checks if the malicious app version calls other sensitive APIs but is not called by your benign version. If this happens, we consider that the test generation tool under analysis could detect the specific malware.

In the final process, we generate a report that includes a set of observations like the tool name, the number of the repetition (in the range [1..3]), and a boolean value indicating whether or not the malware has been identified. Figure~\ref{fig:accuracy} presents the result of each individual test generation tool in detecting malware at our data set (824 app pair).

Among the malware detected by all test tools, we also check its similarity score. We choose does not to discard any malware based on a selected threshold, since this could affect both false positives and false negatives. For example, if we use a high similarity score for select malware, we will have a low false positive and high false negative. Otherwise, a low similarity score brings high false positives and low false negatives. Zhou et al.\cite{DBLP:conf/codaspy/ZhouZJN12} mention the threshold of 70\% as a good balance to infer whether an app is repackaged or not. Hence, our results also present how many detected malware are above and below this threshold (70\%), at each test tool.\newline
\newline
\textbf{Droidbot:} When using Droidbot to build sandbox, our results present that it could detect a total of $677$ malware among $824$ app pairs in the dataset (82.16\%). Among tools used in our study, Droidbot was the most efficient in terms of the detection of apps with malicious behavior. At other works~\cite{DBLP:conf/wcre/BaoLL18}\cite{DBLP:journals/jss/CostaMMSSBNR22}, Droidbot has been shown to be the most efficient tool, whose resulting sandbox detected the largest number of malicious apps. Among all malicious apps detected, $347$ have a similarity score below 70\% and $330$ above.\newline
\newline
\textbf{Monkey:} The tool part of the Android SDK produced a sandbox that detected $527$ malware out of the dataset (63.95\%). It created the second best efficient sandbox in our study, although it implements the most basic random strategy test~\cite{DBLP:conf/icst/WetzlmaierRP16}\cite{DBLP:conf/kbse/ChoudharyGO15}. $288$ malware detected has similarity score below 70\% and $239$ above.   \newline
\newline
\textbf{Droidmate2:} Droidmate was designed with an explicit goal of monitored calls to Android sensitive methods based on a set of sensitive APIs defined in the framework AppGuard~\cite{DBLP:conf/esorics/BackesGHMS13}. Although it has this specific goal, it sandbox detected $414$ malicious apps (50.24\%). Only a little more than half of the dataset. Among malware detected, $224$ have a score below 70\% and $190$ above.\newline
\newline
\textbf{Humanoid:}  The tool that emulates realistic users had the worst performance in comparison to others, that rely on random testing (such as Monkey). Since Humanoid create human-like test inputs, based on a learned model, we believe that in a simulated environment, its method to generate test inputs is impaired. The resulting sandbox identified 401 malicious apps in our dataset (48.66\%), less than half of the dataset. Regarding similarity score, in view of threshold (70\%), it was the only tool that have less malware with a score below 70\% among all detected, $197$ malware, and more above 70\%, $204$ malware.

\begin{figure}[ht]
\centering
\includegraphics[scale=0.3]{images/accuracy.pdf}
\caption{Accuracy of each test generator tool on malware detection.}
 \label{fig:accuracy}
\end{figure}

\subsection{Path Analysis}\label{sec:path}

Our investigation into malicious apps was also carried out through a Paths analysis. The analysis is conducted checking the callgraph created by Logcat when performance the app pairs, on each test generator. First, we make a filter to collect just the paths between the app entry point and the call to any method sensitive, based on a methods sensitive list from framework AppGuard.

Finally, with both callgraph, from benign and malicious app versions, we compare them. If we spotting differences between any paths, means that the sensitive methods are accessing in a different way from app pairs, and we can consider it as malware.

Because all app pairs detected at our first study, using Mine Sandbox, have naturally different paths, we choose to investigate only app pairs that were not detected at Mine Sandbox approach. Our results show that it is possible to improve malware detection if we also consider the path between app entry point and sensitive methods call.

Table~\ref{tab:pa} summarizes the results of this investigation. The column Execution (NID) shows the number of malware not identified when executing each tool, at mine Sandbox approach. Column  Path Different (PD) shows the number of app pairs, among no identified at first columns, have different paths from the entry point to the sensitive method call. Improve column shows(in percentage) how many the path analysis could improve malware detection if this approach is also considered. We calculate the improvement using Eq.~\eqref{improve}


\begin{table}[ht]
  \caption{Summary of the results of path analysis. }
  \centering
  \begin{small}
 \begin{tabular}{lrrr}
   \toprule
   Tool & Execution (NID) & Path Analysis (PD) & Improve (\%) \\   \midrule
   Monkey &  297 & 157 & 52.86 \\ 
   Droidmate &  410 & 259 & 63.17 \\ 
   Droidbot &  147 & 103 & 70.06 \\ 
   Humanoid &  423 & 27 & 39.71 \\ 
 \bottomrule
 \end{tabular}
 \end{small}
 \label{tab:pa}
\end{table}

\begin{eqnarray}
Improve & = & \frac{Path Analysis (DP) \times 100}{Execution (ND)} 
\label{improve}
\end{eqnarray}

As we can see in Table~\ref{tab:pa}, although Droidbot already had the best accuracy at first study, if we also consider compare paths, Droidbot could benefit the most from working together with this approach. On the other hand, Humanoid that does not identify 423 app pair, the worst performance at first study, do not have good support of the compare path proposal, like other tools, having the accuracy improving at just 39.71\%.

\subsection{Manifest File Analysis}\label{sec:manifest}

We also check some particulars from Manifest file, that point to a likely suspicious behavior, and also could give support to mine Sandobox approach. In section \ref{sec:manifestAnalysis}, we show that an automatic hacking script could inject duplicated permission and actions at Manifest file. We investigated if this happened at pair no detected at first study. We also check if among these pairs, there were requests to new permissions, that were not initially requested by benign version, or if there was an excessive request from permissions at malicious version. Table~\ref{tab:mfa} gives some features of Manifest file from malware not detected at first study. 

\begin{table}[ht]
  \caption{Manifest File with duplicate code.}
  \centering
  \begin{small}
 \begin{tabular}{lccccc}
   \toprule
   Tool & (NID) & (DP) & (DA) & (DP or DA) & (DP and DA) \\   \midrule
   Monkey &  297 & 36 & 44 & 62 & 18 \\ 
   Droidmate &  410 & 48 & 63 & 91 & 20 \\ 
   Droidbot &  147 & 18 & 25 & 36 & 7 \\ 
   Humanoid &  423 & 51 & 61 & 91 & 21 \\ 
 \bottomrule
 \end{tabular}
 \end{small}
 \label{tab:mfa}
\end{table}

Column (NID) indicates the number of malware not detected at first study. It is the same numbers that have been reported at table~\ref{tab:pa} (Execution NID). The second column (DP) indicates how many Manifest files had duplicated permission, among malware at first column (NID). Column (DA) follows the same idea as the previous column, presenting now the duplicated actions. The last two columns present the combination of columns (DP and DA) by logic operators (And/Or).

We deduce that if we have on the same Manifest file, duplicate request permissions, and actions, it was not injected by mistake, and it is a clue that the Manifest file was modified for malicious purposes. Considering this view, $18$ among app pairs do not detect by Monkey tool, had this feature, confirming that they were indeed malware. The same feature we find at $20$ app pair, that was not detected by sandbox build by Droidmate test tool, $7$ and $21$ app pairs by Droidbot and Humanoid respectively. Hence, Humanoid tool had more app pairs with Manifest files with a strong suspicion of malicious modify, among app pairs do not detected by its sandbox. 

Interestingly, among all app pairs at last columns (69 app pairs), just $4$ have the similarity score above 70\%, $2$ from Monkey, $1$ from Droidmate, and $1$ from Humanoid. Other app pairs with very suspicious Manifest file, have a high similarity score, which indeed makes it difficult detection with mine Sandbox approach.

At final analysis, we investigate how many app pairs request more permission than your benign version, featuring a suspicious amount of permission requests. Some works present that a benign Android app normally requests on average $4$ permission, while a malicious apps version requests a median number of $7$ permission\cite{DBLP:conf/soups/FeltHEHCW12}\cite{DBLP:journals/tifs/0029LBKTLC17}. Hence, and our study we investigate how many app pairs have at least $3$ more permission than its benign version. Table~\ref{tab:mp} presents the results.

\begin{table}[ht]
  \caption{Manifest File with more permissions.}
  \centering
  \begin{small}
 \begin{tabular}{lccc}
   \toprule
   Tool & (MP) & (DP) & (NDP) \\   \midrule
   Monkey &  10 & 8 & 2 \\ 
   Droidmate &  11 & 8 & 3 \\ 
   Droidbot &  3 & 1 & 2 \\ 
   Humanoid &  15 & 11 & 4 \\ 
 \bottomrule
 \end{tabular}
 \end{small}
 \label{tab:mp}
\end{table}

The second column (MP) present how many app pairs, no detected by each test tool, have other suspicious feature of Manifest file. It has request at least $4$ or more permission than its benign version. Third and fourth columns (DP)(NDP) presents how many app pair from second columns have duplicated permission or not respectively. As we can see, most of app pair that have excessive permission request, also have duplicated permission in your code. From table~\ref{tab:mp}, we note that test tool Humanoid, has the most apps no detected by mine Sandbox approach with excessive request permission.


\subsection{Implications}\label{sec:implications}
\section{Exploring the MAS Extensions}\label{sec:complementary}


%Our work, although closely related to previous studies, differs from them in several aspects.  First, our assessment is more comprehensive: instead of considering $102$ pairs of benign/malign apps, we execute our study considering \apps pairs of apps. We then investigate which characteristics of the malware samples in the large dataset explain the lower performance.

We extend our investigation by evaluating two extensions to the \mas. The first extension leverages dynamic call graphs
to analyze the paths from entry points to calls made to sensitive APIs (in both the original and repackaged versions of an application).
An inspiration for this extension comes from the work of Jamrozik et al.~\cite{DBLP:conf/icse/JamrozikSZ16}.
The second extension enhances the \mas by comparing the values of actual parameters used in the calls to sensitive APIs,
drawing inspiration from the work of Le et al.~\cite{le2018towards}. We conduct a series of experiments using our \cds,
employing three configurations of the \mas: (a) with the call-graph-based extension enabled,
(b) with the parameter-based extension enabled, and (c) with both extensions enabled.
So, in this section we explore an additional research question: (RQ4) \rqe.

\subsection{Call Graph Path Analysis}

Regarding the first extension (call graph path analysis), our initial step involves constructing a
dynamic call graph that encapsulates the execution flow of the Android apps,
distinguishing between the original and repackaged variants. We derive two distinct sets:
one comprising all paths from entry points to calls invoking sensitive APIs identified in the dynamic call graph generated during the execution of
DroidBot on the original app version (designated as set $S_{p1}$), and repackaged app version (designated as set $S_{p2}$).
Following that, we calculate the set difference denoted as $S_p = S_{p2} \setminus S_{p1}$.
In cases where $S_p$ is not empty, we categorize the repackaged version as potential malware.
This extension complements a variant of the \mas proposed by Jamrozik et al.~\cite{DBLP:conf/icse/JamrozikSZ16},
which considers not only the calls to sensitive APIs but also the events that trigger those calls.
Our underlying assumption is that the integration of path analysis can enhance the effectiveness of
the \mas in the realm of malware classification.
In Figure~\ref{fig:callGraph} , both the original and repackaged
versions invoke the same sensitive method, \texttt{getSubscriberId()}.
This method retrieves the unique subscriber ID of the device and necessitates the manifest file
permission \texttt{READ\_PHONE\_STATE}, which is present in both versions of the app.
Notably, the original app calls this method through two distinct paths (\emph{Path 01} and \emph{Path 02}),
indicating an expected user action. However, in addition to the two original paths,
the repackaged version introduces a third path (\emph{Path 03}). This new path initiates
from a method that conducts a stealth computation on a background thread, suggesting an action that could
commence without the user's awareness.


\begin{figure}[ht]
\centering
\includegraphics[scale=0.30]{images/maliciousCallGraph.pdf}
\caption{Illustrative example of the path analysis. In this case, both versions call the same set of sensitive APIs. Nonetheless,
the paths between the entry points and the calls to sensitive APIs diverge.}
 \label{fig:callGraph}
\end{figure}

\begin{figure}
\centering
\includegraphics[scale=0.28]{images/maliciousTrace_example01.pdf}
\caption{Example of suspect path.}
 \label{fig:maliciousTrace}
\end{figure}

Recognizing that a malicious repackaged version of an app may exhibit the same set of calls
to sensitive APIs as the original, albeit with additional paths, we conducted an additional experiments
using all app pairs in our \cds. Our aim was to assess whether this path-based analysis extension could enhance
the accuracy of the \mas for malware classification. Our findings indicate that the inclusion of path analysis
leads to a modest reduction in false negatives, though, decreasing from 1,306 to 1,172.
However, this improvement comes at the expense of a slight increase in false positives, rising from 610 to 619
(as depicted in the first and third rows of Table~\ref{tab:accuracyParameter}).
In conclusion, the extension of path analysis enhances the overall accuracy ($F_1$) of the vanilla
\mas from 0.59 to 0.63.

\subsection{Parameter Analysis}

As described in~\cite{le2018towards}, malicious apps might require external data,
such as a remote server address to push advertisements, or sink sensitive information to another location,
different from the original version, using SMS messages (for instance). For this purpose, malicious apps
might change the actual parameter values used in calls to sensitive APIs, values that might differ from those
used in the original app. Thus, we hypothesize that differences on the parameter values (from both app versions), passed to the same calls to
sensitive APIS, might provide hints of suspiciously repackaged apps. Figure~\ref{fig:parameterDiff} presents a concrete example of
different parameter values used in the calls to the same sensitive method. This example comes from the \texttt{com.nla.downloader} app
(one of the apps in the \cds), and uses a \texttt{java.net.URL} object to remunerate a different publisher from the original app, an expressly harmful behavior. We executed our
experiment one more time, to estimate the impact of parameter analysis into the accuracy of the \mas for
malware detection. 
The parameter analysis reduces the number of false negatives from 1,306 (vanilla \mas) to 1,221, with the side effect of
increasing the number of false positives from 610 to 621 (see the first and second rows of Table~\ref{tab:accuracyParameter}).
In general, the accuracy ($F_1$) of the \mas using parameter analysis improves from 0.59 to 0.61 at \cds.

\begin{figure*}[t]
\centering
\includegraphics[scale=0.3]{images/parameterDiff.pdf}
\caption{Example of different parameter values used in the same method call.}
 \label{fig:parameterDiff}
\end{figure*}

\begin{table*}[ht]
  \caption{Accuracy of the \mas with aid of complementary techniques (3,211 app pairs).}
\centering{
  \begin{tabular}{lrrrrrr} \hline
    Dataset & TP   & FP  & FN  & Precision & Recall & $F_1$ \\ \hline
    
    %\mas + Traces  & \sds (102)   & 67   & 18  & 2   & 0.78      & 0.97   & 0.87  \\
    \cds \mas    & 1,423  & 610 & 1,306 & 0.69       & 0.52   & 0.59  \\
    \cds \mas + Trace     & 1,557  & 619 & 1,172 & 0.71       & 0.57   & 0.63  \\
    \cds \mas + Parameter     & 1,499  & 621 & 1,221 & 0.70       & 0.55   & 0.61  \\
    \cds \mas + Trace + Parameter     & 1,622  & 630 & 1,099 & 0.72       & 0.59   & 0.65  \\
    %\mas + Traces  & \cds (1203)   & 214  & 326 & 245 & 0.39      & 0.46   & 0.42  \\ 
    \hline
  \end{tabular}
  }
  \label{tab:accuracyParameter}
\end{table*}

\subsection{Combining The Path and Parameter-based extensions} \label{sec:combination}

Finally, we investigate if we could benefit
from combining both extensions (path and parameter-based extensions).
The execution that combines both extensions correctly classifies 1,622 repackaged apps as malware (TP),
reducing the number of FN from 1,306 to 1,099. Nonetheless, this execution increases the number of FP from 610 to 630.
The combination of both techniques proves to be more effective than the vanilla \mas or when we use just one of the extensions.
In summary, the results of our last experiment reveal that the combination of both techniques presents an $F_1$ score of
0.65 (see Table 6).

\tb{5}{When combining both techniques, we reduces the number of false negatives (in comparison with the vanilla \mas), with the side effect of increasing the number of false positives. However, this configuration improves the
overall accuracy ($F_1$) of \mas at malware detection, from 0.59\% to 0.65\% at \cds.} 

%\input{newDataset}
\section{Discussion}

In this section, we explicitly answer our research questions,
summarize the implications of our results, and discuss possible
threats to the validity of the results presented so far.

\subsection{Answers to the Research Questions}

The results we presented in the previous sections
 answer our four research questions, as
we summarize in the following.

\begin{itemize}
\item \textbf{Performance of the \mas on the \cds (RQ1).} 
  Our study indicates that the accuracy of the \mas reported in previous studies~\cite{DBLP:conf/wcre/BaoLL18,DBLP:journals/jss/CostaMMSSBNR22} does not
  generalize to a larger and more diverse dataset. That is, while in our
  reproduction study (using the \sds of previous research) the vanilla \mas
  leads to an accuracy of 0.89, we observed a drop of precision and recall
  that leads to an accuracy of 0.42 in the presence of our \cds (\apps pairs of
  original and repackaged versions of Android apps). 

\item \textbf{Effect of trace analysis (RQ2).} We do not find any gain of enriching the vanilla \mas with
  trace analysis in terms of accuracy (F-score). Although the use of traces
  reduces the number of false negatives (improving recall), it also increases the number of false
  positives with a similar proportion---which does not change the $F_1$ measure significantly.
  Nonetheless, for the context of malware identification, we believe that the
  gain in recall might justify the use of trace analysis.

\item \textbf{Similarity Analysis (RQ3).} Our results bring evidence about the existence of a negative
  association between similarity and the ability of the \mas to correctly
  classify a repackaged version of an app as a malware. Nonetheless,
  our similarity analysis alone was not sufficient to explain the low
  performance of the \mas to identify malware in the \cds.

\item \textbf{Malware Family Analysis (RQ4).} The results indicate that a specific family
  (\gps)  is responsible for the largest number of false
  negatives in the complete dataset. The \gps family corresponds to a particular type of
  Adware, designed to automatically display advertisements while an app is executing. After reverse engineering
  a sample of 20 \gps malware, we confirmed that the vanilla \mas and its trace variant cannot identify the
  patterns of changes introduced in the repackaged versions of the apps. The \gps family
  is the Achilles' hell of the \mas, since its accuracy increases substantially
  when we remove the \gps samples from our \cds.  
\end{itemize}


\subsection{Implications}\label{sec:implications} 

Contrasting to previous research works~\cite{DBLP:conf/wcre/BaoLL18,DBLP:conf/iceccs/LeB0GL18,DBLP:journals/jss/CostaMMSSBNR22},
our results 
%discussed in the previous sections 
lead to a more systematic understanding
of the limitations of using the Android mining sandbox approach
for malware detection. In particular, we show that,
in the presence of a large dataset, the performance of the
approach drops significantly. 

We also highlight that considering only the differences in the
sets of calls to sensitive APIs also leads to false negatives. We
argue in favor of using a more elaborate \emph{diff} approach, which
extends the mining sandbox approach to include the comparisons of
the dynamic call traces from the ``entry points'' of an Android app to its
set of calls to the sensitive APIs. We collect these call traces while
building the sandbox in the usual way~\cite{DBLP:conf/icse/JamrozikSZ16}. Using this extension, we could improve
the performance of the vanilla mining sandbox approach by a factor
between 19\% (small dataset) to 22\% (complete dataset). We
also explored a more ``complementary'' approach, somewhat independent
of the mining sandbox approach, which statically search for two
specific patterns of changes in the Android manifest files: permission
duplication and component duplication. Using the manifest analysis,
we complement the performance of the mining sandbox approach, improving its performance by
a factor between 15\% (complete dataset) and 17\% (small dataset).
Altogether, the implications of this research are twofold:

\begin{description}
  \item[A warning to the community:] the mining sandbox approach for malware detection exhibits a much higher false negative rate  than previous research reported. 
  \item[Future directions:] researchers should advance the mining sandbox approach for malware detection by exploring more advanced techniques for differentiating benign and malicious versions of the apps. 
\end{description}  


\subsection{Threats to Validity}\label{sec:threats}

% \todo[inline]{MM: Needs major revision.}

There are some threats to the validity of our results.
Regarding {\bf external validity}, one concern relates to the 
representativeness of our malware datasets and how generic our findings are.
Indeed, mitigating this threat was one of the motivations for our research,
since, in the existing literature, researchers had explored just
one dataset of 102 pairs of benign/malign apps. Curiously,
for this small dataset, the performance of the
mining sandbox approach is more than twice superior
than its performance on our complete dataset (800 pairs of
apps). We contacted the authors of the Bao et al. research paper, asking them
if they have used any additional criteria for selecting the
dataset. Their answers suggest the contrary: they have not used
any particular app selection process that
could explain the superior performance of the mining
sandbox approach for the small dataset. We believe that
our results in the complete dataset (800 apps) generalize better than previous research works,
since we have a more comprehensive collection of malwares with different
categories and degrees of similarity. Nonetheless, our
research focus only on Android repackaged malwares and thus we
cannot generalize our findings to malwares targeting
other platforms or that use different approaches to
generate a malicious asset.

During the exploratory phase of the mining sandbox approach,
we collected the set of calls to sensitive APIs the benign version of
an app executes, while running a test case generation tool (such as
DroidBot). That is, the mining sandbox approach assumes the existence of a true benign
version of a given app in the exploratory phase. Regarding {\bf conclusion validity}, a
possible threat to our research is the use of \emph{pseudo-benign} versions
of some apps, which might compromise the results of our research.
To mitigate this threat, we query the VirusTotal repository to search for the
status of the benign and malicious versions of 100 apps,
randomly selected in our complete dataset. Among the benign
versions of these 100 apps, 14 apps were classified as a malware 
by at least two antivirus tools. When considering
the malicious version of these 100 apps, 44 apps were
classified as a malware by at least two antivirus tools. This might indicate
a possible misclassification in the AndroZoo repository---or a need
for an update. 

\begin{comment}
    
Regarding the \textbf{correlation between dataset properties and accuracy drop}, after running statistical tests (logistic regression),
we could not find evidence that the \emph{diversity} of the
complete dataset---in terms of similarity score and types of malware-
is responsible for the higher number of false negatives of the mining
sandbox approach. This implies that there was no 1-1 correlation between the brackets of similarity index, malware types to the drops in accuracy. Therefore, further research is necessary to investigate
other possible reasons for that. Perhaps, the complete dataset
contains a large percentage of malware that use more
advanced techniques to evade from both static and dynamic analysis---
both methods are used in the mining sandbox approach
we discussed in this paper.


\end{comment}



Regarding {\bf construct validity}, we address the main threats
to our study by using simple and well-defined metrics that are
in use for this type of research: number of correctly classified
malwares in a dataset (true positives) and the number of assets that the
classification approaches we explored here fail to
correctly classify (false negatives). Based on these metrics we computed
the recall of the approaches. We did not consider the
metric precision here, because the ``malicious'' assets in our
dataset are all true positives---hence precision would
always be of 100\%. In a preliminary study, we
investigated whether or not the mining sandbox approach
would classify a benign version of an app as a malware,
after executing the test case generation tools in multiple
runs. After combining three executions in a benign version to
build a sandbox, we did not find any other execution that
could wrongly label a benign app as a malware. So we are
confident that the mining sandbox approach leads to a
high precision, even though it has shown a poor recall
in our complete dataset. 


%\section{Conclusions and Future Work}\label{sec:conclusions}
\section{Conclusions and future work}\label{sec:conclusions}

The Mining Android Sandboxes (MAS) approach~\cite{DBLP:conf/icse/JamrozikZ16}
has been tailored for Android malware detection~\cite{DBLP:conf/wcre/BaoLL18}
and empirically validated in a couple of studies~\cite{DBLP:conf/wcre/BaoLL18,DBLP:conf/iceccs/LeB0GL18,DBLP:journals/jss/CostaMMSSBNR22}.
To better understand the strengths and limitations of the \mas for malware detection,
this paper reported the results of an empirical study that reproduces previous research
work~\cite{DBLP:conf/wcre/BaoLL18,DBLP:journals/jss/CostaMMSSBNR22} using a larger and more
diverse dataset---comprising \apps pairs of \emph{original} and \emph{repackaged} apps.
To our surprise, compared to results already published,
the performance of the \mas drops significantly in this new dataset, mainly
because the \mas fails to detect a popular family of Android malware (named \gps). 
We also evaluated an extension to the \mas that we designed to improve the overall
accuracy of the approach for malware detection.
Although this extension reduces the number of false negatives of the \emph{vanilla \mas}, it was not
sufficient to increase the \mas accuracy in the large dataset. These negative results
brought evidence of the need to complement the \mas with other techniques, so that it could be
effective for Android malware detection.

\balance 

\bibliographystyle{IEEEtran}
\bibliography{ref}








\end{document}


