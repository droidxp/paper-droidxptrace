\section{Conclusions and Future Work}\label{sec:conclusions}

In this paper, we conducted a study in which we observed the accuracy result, in terms of malware detection, of the state of the art mining sandbox approach. In our experiment, we used the test generator tool Droidbot to explore sensitive APIs called by malware at a real-word dataset of $800$, containing more representative samples than previous works. Our results demonstrated that the accuracy of Droidbot for malware detection drops to $24.12\%$ when we compare it with previous works, which explored a smaller sample. This first experiment also helps us to conclude that only a few sensitive APIs are responsible for most injected malware code at our repackaged sample.

These results encourage us to investigate how we could improve the mine sandbox approach. Hence, we investigated how relevant a dynamic call graph analysis and a simple manifest files analysis are to improving malware detection in the mine sandbox approach. Our investigation present relevant findings. First, with dynamic call graph analysis, we find that its possible to improve malware detection, if we consider trace analysis as a factor. Moreover, we present that a simple static analysis of manifest files also improves malware detection, complementing even the dynamic call graph analysis in terms of malware detection.

Our finds provide possible future research directions to further improve the mine sandbox approach. As future work, we plan to investigate if the distance at a dynamic call graph, between the Android app entry point and a call for a sensitive API has influence on the accuracy rate of malware detection at mine Android sandbox. We also plan to investigate the hypothesis that the test tool coverage used at mine Android Sandbox, can further benefit the approach since it will provide more traces at dynamic call graph to inspect.

