%\section{Conclusions and Future Work}\label{sec:conclusions}
\section{Conclusions and future work}\label{sec:conclusions}

The Mining Android Sandboxes (MAS) approach~\cite{DBLP:conf/icse/JamrozikZ16}
has been tailored for Android malware detection~\cite{DBLP:conf/wcre/BaoLL18}
and empirically validated in a couple of studies~\cite{DBLP:conf/wcre/BaoLL18,DBLP:conf/iceccs/LeB0GL18,DBLP:journals/jss/CostaMMSSBNR22}.
To better understand the strengths and limitations of the \mas for malware detection,
this paper reported the results of an empirical study that reproduces previous research
work~\cite{DBLP:conf/wcre/BaoLL18,DBLP:journals/jss/CostaMMSSBNR22} using a larger and more
diverse dataset---comprising \apps pairs of \emph{original} and \emph{repackaged} apps.
To our surprise, compared to results already published,
the performance of the \mas drops significantly in this new dataset, mainly
because the \mas fails to detect a popular family of Android malware (named \gps). 
We also evaluated an extension to the \mas that we designed to improve the overall
accuracy of the approach for malware detection.
Although this extension reduces the number of false negatives of the \emph{vanilla \mas}, it was not
sufficient to increase the \mas accuracy in the large dataset. These negative results
brought evidence of the need to complement the \mas with other techniques, so that it could be
effective for Android malware detection.
