\subsection{Threats to Validity}\label{sec:threats}

% \todo[inline]{MM: Needs major revision.}

There are some threats to the validity of our results.
Regarding {\bf external validity}, one concern relates to the 
representativeness of our malware datasets and how generic our findings are.
Indeed, mitigating this threat was one of the motivations for our research,
since, in the existing literature, researchers had explored just
one dataset of 102 pairs of benign/malign apps. Curiously,
for this small dataset, the performance of the
mining sandbox approach is more than twice superior
than its performance on our complete dataset (800 pairs of
apps). We contacted the authors of the Bao et al. research paper, asking them
if they have used any additional criteria for selecting the
dataset. Their answers suggest the contrary: they have not used
any particular app selection process that
could explain the superior performance of the mining
sandbox approach for the small dataset. We believe that
our results in the complete dataset (800 apps) generalize better than previous research works,
since we have a more comprehensive collection of malwares with different
categories and degrees of similarity. Nonetheless, our
research focus only on Android repackaged malwares and thus we
cannot generalize our findings to malwares targeting
other platforms or that use different approaches to
generate a malicious asset.

During the exploratory phase of the mining sandbox approach,
we collected the set of calls to sensitive APIs the benign version of
an app executes, while running a test case generation tool (such as
DroidBot). That is, the mining sandbox approach assumes the existence of a true benign
version of a given app in the exploratory phase. Regarding {\bf conclusion validity}, a
possible threat to our research is the use of \emph{pseudo-benign} versions
of some apps, which might compromise the results of our research.
To mitigate this threat, we query the VirusTotal repository to search for the
status of the benign and malicious versions of 100 apps,
randomly selected in our complete dataset. Among the benign
versions of these 100 apps, 14 apps were classified as a malware 
by at least two antivirus tools. When considering
the malicious version of these 100 apps, 44 apps were
classified as a malware by at least two antivirus tools. This might indicate
a possible misclassification in the AndroZoo repository---or a need
for an update. 


Regarding {\bf construct validity}, we address the main threats
to our study by using simple and well-defined metrics that are
in use for this type of research: number of correctly classified
malwares in a dataset (true positives) and the number of assets that the
classification approaches we explored here fail to
correctly classify (false negatives). Based on these metrics we computed
the recall of the approaches. We did not consider the
metric precision here, because the ``malicious'' assets in our
dataset are all true positives---hence precision would
always be of 100\%. In a preliminary study, we
investigated whether or not the mining sandbox approach
would classify a benign version of an app as a malware,
after executing the test case generation tools in multiple
runs. After combining three executions in a benign version to
build a sandbox, we did not find any other execution that
could wrongly label a benign app as a malware. So we are
confident that the mining sandbox approach leads to a
high precision, even though it has shown a poor recall
in our complete dataset. 

