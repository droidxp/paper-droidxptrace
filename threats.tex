\section{Threats to Validity}\label{sec:threats}

\todo[inline]{MM: Needs major revision.}

There are some threads to the validity of our results. One of them is the representativeness of our malware dataset and how generic was our findings. To care out of this threat, our data set is based on a generic and large set of Android apps. We incorporate more $699$ app pair than previous studies from different categories, and with an average similarity coefficient smaller than these studies. Thus, our findings based on malware could be more generalized to real-world malware. In a future work, we can further mitigate this threat incorporating more app pairs.

A second threat is related to correctness and result analysis. To minimize the threat and validate our findings, we carefully setup our infrastructure following close configuration of previous works. Before the deployment, we selected the test generator tool more adequate, the execution time and execution numbers enough. In the end, we conduct a manual observation of results on a significant sample, where we can confirm our observations. Finally, the final results also were cross-checked by others co-authors. Future researches are encouraged to replicate our experiment, since we also made our scripts and results publicly available~\footnote{ https://github.com/droidxp/paper-droidxptrace-results}.