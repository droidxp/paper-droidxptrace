\section{Related Work}\label{sec:relatedwork}
In this section, we discuss prior studies in two areas: Dynamic and Static Analysis on Android apps and mine Sandbox.

\subsection{Dynamic and Static Analysis on Android apps}\label{sec:analysis}

There is a large body of work that explores the use of program analysis techniques to detect malware. 

Several works have been proposed to detect malware based on sensitive method calls and permission control~\cite{DBLP:conf/mobicom/WeiGNF12,DBLP:conf/asiajcis/WuMWLW12,DBLP:conf/sp/LiDLDG21}. Cai et al.~\cite{DBLP:journals/tse/CaiR21} presented a longitudinal study on Android apps focusing on run-time behaviors. However, this work does not focus specifically on malware detection but on general security gaps in apps by considering only benign apps. Fangfang et al.~\cite{DBLP:conf/wisec/ZhangHZW014} proposed ViewDroid, which models the UIs of Android apps as a directed graph. Although ViewDroid also works by comparing app pairs to identify repackaged apps, their focus is UI centric.

On static analysis approaches exploring Android Manifest files, Kim et al. proposed RomaDroid~\cite{DBLP:journals/access/KimLCP19}.  Their approach does not consider the structural context in Manifest files, but rather treat the files as sequence of strings and perform a lowest common subsequence (LCS) based approach to detect repackaged apps. Au el al.~\cite{DBLP:conf/ccs/AuZHL12} also apply static analysis on Android Manifest files to detect vulnerabilities in Android apps. They do this by mapping requested permissions to sensitive API calls in the code.

Li et al.~\cite{DBLP:journals/tifs/0029LBKTLC17} provided a systematic knowledge on Android malware by conducting an empirical study comparing malicious repackage app with their benign counterparts (1,497 app pairs). They found that the majority of Android malware are repackaged versions of benign apps that donot do anything complex modifications, many times simply reusing library code.

 In the domain of detecting repackaged apps by comparing app pairs, Crussell et al.~\cite{DBLP:conf/esorics/CrussellGC12} proposed  DNADroid, which compares program dependence graphs, and Zhou et al.~\cite{DBLP:conf/codaspy/ZhouZJN12} DroidMoss which detects and analyzes repackaged apps using a fuzzy hashing technique.

\kn{We should add some lines here why we did not use any of these approaches and why our list Droidbot etc. is different from there. Maybe mention something in the lines of "How state of the art our tools are"?}

\subsection{Mine Sandbox}\label{sec:mineSandbox}

