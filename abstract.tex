\IEEEtitleabstractindextext{
\begin{abstract}
Android is the most popular operating system for the mobile platform, and smartphones’ ubiquitous nature in our daily lives has made their security an important concern for researchers and practitioners alike. Among other techniques, previous research results have advocated using the Mining Android Sandbox approach (\mas) to identify malicious behavior in repackaged Android applications (apps), one recurrent method to inject malicious behavior. Nonetheless, these previous studies have drawn their conclusions using a small dataset of \appsSmall pairs of original and repackaged apps, threatening the findings w.r.t. external validity and opening the question of whether or not the MAS approach scales to larger datasets. To shed light into these issues, in this paper we report the results of an experiment that reproduces the state-of-the-art research on empirically evaluating the MAS approach accuracy. Differently from previous research work, our study uses a dataset that is an order of magnitude larger than the datasets used in previous research (a total of \apps pairs of apps with a much diverse malware families). To our surprise, our results reveal that the \mas accuracy for malware identification drops significantly: $F_1$ score drops from \fscoreSmall in the previous studies to \fscore in our larger dataset. After an in-depth assessment, we found that the representative number of malware from the \gps family \alert{partially} explains the higher number of samples for which the \mas fails to correctly classify a repackaged version of an app as a malware. \review{We also investigate the benefits of two extensions we implemented for the \mas: Trace Analysis and Parameter Analysis. The extensions bring a only slight improvement to the \mas performance: enabling both extensions improves the accuracy of the vanilla \mas in 5\% (from 60\% to 65\%)---still far from the accuracy that previous studies report.} Our findings open the discussion on the possible blind spots that plague the \mas and their accuracy issues when scaled and reveal the need for complementing the \mas with other techniques so that it could effectively detect a broader class of malware. 
\end{abstract}
}

%\begin{IEEEkeywords}
%  Android Malware Detection, Dynamic Analysis, Mining Android Sandboxes
%\end{IEEEkeywords}
