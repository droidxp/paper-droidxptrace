\begin{abstract}

Android is by far the most popular operating system for the mobile platform and the ubiquitous nature of smartphones in our daily lives has only made its security a significant topic for researchers and practitioners alike. Previous research has shown that security experts can benefit from the mining sandbox approach to classify malware. The Android OS exposes several sensitive APIs that allow apps to gain access to user's sensitive resources like contacts, locations and call logs. Nonetheless, in the literature, we did not find a quantitative assessment that characterizes malicious behavior, in terms of calls to sensitive APIs. Moreover, previous research reports that the mining sandbox approach has some limitations (a false negative rate around 35\%).

To complement prior studies and mitigate the limitations of mining sandbox approaches, we address both issues in our paper. First, we carefully investigate the calls to sensitive APIs the malicious apps introduce. From our investigation of $800$ repackaged malicious apps, we found that just $16$ sensitive APIs from a list of $162$ were injected into most repackaged malicious apps. This result might help the research community focus their attention on frequently abused sensitive privileges. Second, we explore two techniques that complement the mining sandbox approach and improve the false negative rates: a dynamic call trace assessment that helps identify suspicious paths from app entry points to sensitive API calls, and a simple, yet effective, analysis of Manifest files. Using these new techniques reduces the number of false negatives by a factor of more than 30\%.

\end{abstract}

