\IEEEtitleabstractindextext{
\begin{abstract}
The widespread use of smartphones in our daily lives has elevated concerns regarding their security among researchers and practitioners.
Particularly, security issues are highly prevalent in Android, the most popular mobile operating system. Previous research has explored
various techniques to address these concerns, including the Mining Android Sandbox approach (\mas), which aims to identify malicious behavior in repackaged Android applications (apps).
However, earlier studies have been limited by small datasets, typically consisting of only \appsSmall pairs of original and repackaged apps.
This limitation raises questions about the external validity of their findings and whether the MAS approach is scalable to larger datasets.
To address these concerns, this paper presents the results of an experiment that replicates state-of-the-art research on evaluating the
accuracy of the \mas. Unlike previous studies, our research employs a dataset that is an order of magnitude larger, comprising \apps
pairs of apps covering a more diverse range of Android malware families. Surprisingly, our findings indicate a significant drop in the accuracy of the MAS approach for identifying malware, with
the \fone decreasing from \fscoreSmall in previous studies to \fscore in our larger dataset. Upon closer examination, \review{we discovered that the higher representation of certain malware families partially accounts for the increased number of instance} where the \mas fails to correctly classify a repackaged app as malware. Additionally, we investigated the impact of two extensions—--Trace Analysis and Parameter Analysis—implemented
from the literature to enhance the \mas. However, these extensions only marginally improved the approach's performance, with both extensions combined increasing accuracy by 8\%
(from 54\% to 62\%)—--still falling short of the accuracy reported in previous studies.
Our findings highlight the limitations of the MAS approach, particularly when scaled, and underscore the importance of complementing it with other techniques to effectively detect a
broader range of malware. This opens avenues for further discussion on addressing the blind spots
that affect the accuracy of the MAS approach.
\end{abstract}
}

%\begin{IEEEkeywords}
%  Android Malware Detection, Dynamic Analysis, Mining Android Sandboxes
%\end{IEEEkeywords}
