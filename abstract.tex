\IEEEtitleabstractindextext{
\begin{abstract}
Android is the most popular operating system for the mobile platform, and smartphones’ ubiquitous nature in our daily lives has made their security an important concern for researchers and practitioners alike. Among other techniques, previous research results have advocated using the Mining Android Sandbox approach (\mas) to identify malicious behavior in repackaged apps, one recurrent method to inject malicious behavior into android apps. Nonetheless, these previous studies have drawn their conclusions using a small and {\color{blue} out of date dataset} of \appsSmall pairs of original and repackaged apps, threatening the findings w.r.t. external validity and opening the question of whether or not the MAS approach scales to larger datasets. To shed light into these issues, in this paper we report the results of an experiment reproducing the state-of-the-art research that empirically evaluated the MAS approach accuracy. Differently of previous research work, our study uses two datasets (\cds and \nds). The first one is an order of magnitude larger than the dataset used in previous research (\sds) (a total of \apps pairs of apps with a much diverse malware families). The second is not so large when compared with \cds, however more updated, covering a total of \napps repacked apps with detection date from 2018 to 2022). To our surprise, our results reveal that the accuracy rate of the \mas for malware identification drops significantly, when compared \sds and \cds, ($F_1$ score drops from \fscoreSmall to \fscore). However the $F_1$ score remains stable when compared with \nds, (from \fscoreSmall to \nfscoreSmall). After an in-depth assessment, we found that the representative number of malware from the \gps family {\color{red}partially} explains the higher number of samples for which the \mas fails to correctly classify as malware at \cds. {\color{blue}We also investigated the accuracy of two extensions for the \mas: Trace analysis and parameter analysis. The studies demonstrate that when we combining both extensions we improve the accuracy in 10\% at \cds (\fscore to \fscoreC) and just 2\% at \nds (\nfscoreSmall to \nfscoreSmallC)}, still far from the accuracy that previous studies reported. Our findings open the discussion on the possible blind spots that plague the \mas and their accuracy issues when using diverse datasets, and reveal the need for complementing the \mas with other techniques that it could effectively detect a broader class of malware.
\end{abstract}
}

%\begin{IEEEkeywords}
%  Android Malware Detection, Dynamic Analysis, Mining Android Sandboxes
%\end{IEEEkeywords}