\begin{abstract}
  %Context
  Android is the most popular mobile platform operating system, and smartphones' ubiquitous nature in our daily
  lives has only made their security an important topic for researchers and practitioners alike.
  %Problem
  Previous research results have advocated using the Mining Android Sandbox Approach (\mas) to identify malicious behavior in Android repackaged apps.
  Nonetheless, these previous studies draw their conclusions using a small dataset, threatening the findings concerning external validity and opening the question of whether
  or not the \mas scales for larger datasets.
  %Research Method
  To investigate these issues, we conduct a new experiment using
  (a) a methodology that is similar to that one used in previous research that empirically evaluates the \mas performance and
  (b) a dataset one order of magnitude higher than the datasets used in previous research.
  %Results
  To our surprise, our experiments revealed that the accuracy rate of the \mas for malware identification drops significantly: F-Score drops from 0.89 in the previous dataset to 0.42 in our large dataset.
  After an in-depth assessment, we find that the more significant number of malware from the gappusin family explains the higher number of samples for which the \mas fails to classify as malware correctly (false negative).
  Our findings open the discussion on the possible blindspots that plague the \mas and their accuracy issues when scaled. 
\end{abstract}

\begin{IEEEkeywords}
  Android Malware Detection, Dynamic Analysis, Mining Android Sandboxes
\end{IEEEkeywords}
