\section{Introduction}\label{sec:introduction}

Mobile technologies like smartphones and tablets have become fundamental to the way we function as a society. Almost two-thirds of the world population
uses mobile technologies~\cite{Comscore,DBLP:journals/tse/MartinSJZH17}, with the
Android Platform dominating this market accounting for more than 70\% of the \emph{mobile
market share} with almost 3.5 million Android applications~\footnote{In this paper, we will use the terms Android Applications, Android Apps and Apps interchangeably, to refer to Android software applications} (apps)
available on the Google Play Store~\cite{Statista}. 
With increased popularity also comes increased risk of attacks---motivating plenty of research efforts to design and develop new techniques
to identify malicious behavior or vulnerable code in Android apps~\cite{10.1145/3017427}.


For instance, the
mining sandbox approach takes advantage of automated test case generation tools 
to explore the behavior of an app---in terms of calls to sensitive APIs---and then
generates an Android sandbox~\cite{DBLP:conf/icse/JamrozikSZ16}. During a normal
execution of the app, the sandbox might block any call to a sensitive API
that had not been observed during the exploratory phase. 
Researchers have shown that the mining sandbox approach is also effective
in detecting a popular class of Android malware that ``repackages'' benign apps~\cite{DBLP:conf/wcre/BaoLL18,le2018towards}---i.e., starting with a benign version of an app from an official app store, such as Google Play, one might infect it with
malicious code, such as broadcasting
sensitive information to a private server~\cite{DBLP:journals/tse/LiBK21}. The infected app is then
shared with users using different app stores. 
Previous research works~\cite{DBLP:conf/wcre/BaoLL18,DBLP:journals/jss/CostaMMSSBNR22} compare the accuracy of Android sandboxes for malware detection 
produced from the execution of different test case generation tools, including Monkey~\cite{Monkey}, Droidbot~\cite{DBLP:conf/icse/LiYGC17}, and Droidmate~\cite{DBLP:conf/kbse/BorgesHZ18} tools.
These reports bring evidence that the test generation tool Droidbot outperforms the other tools, leading to sandboxes that
classify as malware 70\% of the repackaged apps in a dataset.

Although revealing promising results, the previous research has two main limitations. First, they use a small dataset of malware comprising only 102 pairs of benign/repackaged versions of an app. Second, their impact analysis is not fine-grained. For instance, they do not consider whether the impact on the accuracy that might be due to malware characteristics, such as the similarity index between the benign and the repackaged version of an app and the malware family (gappusin, kuguo, dowgin, etc.). So, in this paper we present the results of an investigation that aims to replicate previous studies~\cite{DBLP:conf/wcre/BaoLL18,DBLP:conf/scam/CostaMCMVBC20} in a larger dataset of app pairs (benign/repackaged versions), and then explore whether a more diverse sample of app pairs has an influence on the accuracy of the mining sandbox approach for malware identification. To this end, we explore the performance of the mining sandbox approach using a larger dataset we curated for this research and Droidbot~\cite{DBLP:conf/icse/LiYGC17} as test case generation---the tool that, according to the literature, leads to the most accuracy Android sandbox. Our new dataset is an order of magnitude larger (it contains \apps pairs of benign/malicious apps), comprises a much more diverse similarity index, and covers a broader range of malware families. 



{\bf Negative results.} To our surprise, the experimental results show that, on a larger and more diverse dataset (compared to previous studies), the mining sandbox approach's accuracy ($F_1$ score) drops significantly (from 0.89 to 0.42). This result motivated us to conduct a series of experiments to understand the reasons for the lower accuracy in the larger dataset and also explore an extension of the original mining sandbox approach for malware identification. The original approach classifies an app as malware whenever there exists a difference between the sets of calls to sensitive APIs collected when running the test case generation tool over the original and repackaged versions of the same app. In our extension, we also classify as malware a version of an app that calls the same set of sensitive APIs collected in the exploratory phase but with at least one different trace from the app's entry points to the collected sensitive method calls.


Our experiments show that the number of {\bf false negatives} reduces when they are made aware of the
differences in the dynamic call trace, improving the recall 9\% in our large dataset (from 37\% to 46\%). 
Nonetheless, even using the trace analysis extension, the \mas fail to correct classify the samples
from the \gps malware family (a particular class of adware that frequently appears in repackaged apps).
Out of the total of 198 samples within this family, the vanilla \mas and its trace-based variant
fail to correctly classify 171 samples as malware. Our results reveal that this particular family is
responsible for the low accuracy ($F_1 = 0.42$) of the \mas in the large dataset. 
An accuracy of 42\% may still appear unsatisfactory for a truly trustworthy sandbox approach. Still, 
our experiments open the discussion (a) on one important blindspot that must be considered when building 
malware detection approaches that mine sandboxes, and (b) on the need to investigate further sources of
information when trying to distinguish benign and malign app versions.

We detail the \mas and the \mas for malware detection in
Section~\ref{sec:background}. Next, in Section~\ref{sec:experimentalSetup}, we
characterize our experiments in terms of research goal, questions, metrics, datasets, 
and procedures for data collection and data analysis. Section~\ref{sec:results} and
Section~\ref{sec:discussion} detail the main findings of our research and
possible decisions that might threat the validity of our results. Finally,
Section~\ref{sec:conclusions} presents final remarks and possible future
work. The main assets we produced during this research are available in the
paper repository.\footnote{\url{https://anonymous.4open.science/r/paper-droidxptrace-results-F55A/}}
