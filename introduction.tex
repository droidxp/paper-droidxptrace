\section{Introduction}\label{sec:introduction}

Almost two-third of the world use mobile technologies, such as smartphones and tablets, in everyday life in the last decade~\cite{Comscore,DBLP:journals/tse/MartinSJZH17}. 
The Android Operation System dominates this market, with around 3.29 
million Android application~\footnote{In this paper, we will use the terms Android Applications, Android Apps and Apps interchangeably, to refer to Android software applications} (apps) available on the Google Play Store on first quarter of 2022~\cite{Statista}. 
However, as the number of Android apps increases, so does the number of malicious activities and 
malware, which in turn, has lead to the development of many techniques
%This in turn has made security issues in Android apps a relevant research topic. As such, 
to identify malicious behavior or vulnerable code in Android apps.

The focus of this paper is on
%
%One family of approaches
%employ static analysis algorithms to uncover private information 
%leaks, e.g.,~\cite{DBLP:conf/pldi/ArztRFBBKTOM14}, including 
%deobfuscating strings to expose hidden malware~\cite{GlanzMBRAAM20}, 
%\textcolor{red}{\textbf{MM: there is a lot more. I would mention other work, too. ..}}, 
%or misuse of cryptographic primitives.~\cite{DBLP:journals/tse/KrugerSABM21}. 
%An alternative
%%other alternative for 
%protection from Android malicious behavior consists in the 
approaches that use dynamic analysis to mine Android sandboxes~\cite{DBLP:conf/icse/JamrozikSZ16}. 
Specifically, such approaches use automated testing tools 
%(i.e., dynamic analysis) 
to explore apps behavior in terms of access to sensitive resources. Researchers have shown that the mining sandbox approach is also effective in detecting a popular class of Android malware based on repackaging~\cite{DBLP:conf/wcre/BaoLL18,le2018towards}---that is, starting with a benign version of an app from an official app store, such as Google Play~\footnote{\url{https://play.google.com/store/games}}, one might inject it with code performing malicious activities such as broadcasting credit card information to a private server.~\cite{DBLP:journals/tse/LiBK21} %(\kn{Here add a citation that shows prevalence of repackaged malware}).

%The focus of this paper is on malware identification based on the mining sandbox approach: 
In this paper, we (a) assess the state-of-the-art on a larger and more diverse dataset (compared to previous studies),
thereby revealing significant accuracy reduction relative to the what has been reported previously and 
(b) systematically analyse reasons for the observed reduction to provide important insights for 
advancing the state of the art.
%
%Previous work show that 
More specifically, we re-assess DroidBot~\cite{DBLP:conf/wcre/BaoLL18,DBLP:journals/jss/CostaMMSSBNR22}, 
which is the state-of-the-art 
%test-case generation 
tool for malware identification based on the mining sandbox approach,
reported to achieve an accuracy rate between 65\% and 75\%~\cite{DBLP:conf/wcre/BaoLL18,DBLP:journals/jss/CostaMMSSBNR22}. 
 We hypothesize that the relatively high accuracy reported for DroidBot 
 %in earlier evaluations of state of the art mining sandbox approaches for malware detection 
may not generalize to large datasets with a more diverse malware kinds, and a potentially low similarity index.
There are two reasons for our hypothesis. First, DroidBot was evaluated on a rather small set of $102$ app pairs~\footnote{The pairs consist benign and malicious repackaged versions of the same app.} respectively. 
%\textbf{MM: Why two sets? For what respectively?} 
Second, an analysis that we performed revealed that the similarity index between the app pairs used in DroidBot's evaluation was $77\%$, while Bao et al.~\cite{DBLP:conf/iceccs/LeB0GL18} %(\rb{we need to confirm that this is the correct reference. Not sure if (\cite{DBLP:conf/wcre/BaoLL18}) presents a discussion about similarity.}) 
have shown that malware tend to be harder to detect, if the similarity index is lower 
(i.e., in the presence of very dissimilar apps). 

To validate our hypothesis, we study the performance of DroidBot on a more representative dataset. 
%Our dataset is larger 
%dataset with 
%both a more diverse distribution of the similarity index and, on average, with a lower similarity index.
%more diverse distribution of apps in terms of similarity index \textcolor{red}{\textbf{"more diverse similarity index" or "lower similarity index"? kn: it is both more diverse in distribution and on average has a lower similarity index}}. 
Our dataset, which consists of $800$ real app pairs (benign/malicious), is 
roughly 8x larger than the dataset used in the previous study. 
For this reproduction study, we take advantage of DroidXP~\cite{DBLP:conf/scam/CostaMCMVBC20}, a tool suite that supported us not only to integrate DroidBot but also in our study setup and data collection. \textcolor{red}{\textbf{The role of DroidXP needs elaboration.}} The new dataset also has a combined average similarity index of $62.47\%$, which is lower than the original $77\%$, 
as well as a much more diverse distribution of apps in terms of similarity index. 
%\textcolor{red}{\textbf{how did you systematically produce the higher dissimilarity? kn: 
%Our dataset was randomly chosen and the larger dataset automatically introduced many dissimilar app pairs. 

While we were able to reproduce previous results 
on the small dataset of $102$ apps (we achieved a comparable accuracy rate of $63\%$), 
we observed a significant drop of the accuracy of DroidBot on our new dataset of $800$ 
app pairs with a similarity index of $62.47\%$ as much as down to $24.12\%$.
%
%In this paper, we (a) perform a study that confirms this hypothesis and (b) present novel insights to advance the state of the art. 
%Firstly, we assess the accuracy of the state of the art in malware detection using the mining sandbox approaches, DroidBot, on a larger dataset with a more diverse distribution of apps in terms of similarity index \textbf{"more diverse similarity index" or "lower similarity index"?} and observe a significant drop of the accuracy compared to the one reported previously~\cite{DBLP:conf/wcre/BaoLL18,DBLP:journals/jss/CostaMMSSBNR22}. To this end, we present the results of a study where we observed the accuracy results of DroidBot on a set from $800$ real app pairs (benign/malicious). For this reproduction study, we take advantage of DroidXP~\cite{DBLP:conf/scam/CostaMCMVBC20}, a tool suite that supported us not only to integrate DroidBot but also in our study setup and data collection. 
%a lower similarity index
%($800$ as opposed to the $102$). 
%This new dataset also has a combined average similarity index of $62.47\%$ which is lower than the original $77\%$ with a much more diverse distribution of apps in terms of similarity index. 
%To investigate the blindspots that plague the accuracy, 
We hypothesize the source of the problem to be the current technique 
for distinguishing benign from malign app versions employed 
by state of the art as represented by
%the mining sandbox approach for malware detection using 
DroidBot. Existing approaches calculate the
difference between app pairs as
the difference between the sets of sensitive APIs 
that they call. Our work indicates that this characterization may be too simplistic.
More specifically, we identify two 
sources of valuable information for more accurately
distinguishing benign from malign app versions:

\begin{enumerate}
    \item Two apps may call the same set of sensitive APIs, but do so by taking different dynamic call traces between the app's entry point and the sensitive APIs. To investigate possible trace divergences, we used an auxiliary tool from DroidXP project called DroidXPTrace, which compares all traces from an entry point to sensitive APIs call.
    \item The {\color{blue}change patterns} within the Android Manifest' section that requests permissions, when considering the two variants of apps (i.e., benign and malicious variants), may also reveal malware versions. To {\color{blue}mine change patterns} in the Android Manifest files, we harness the standard APK analysis tool from Android~\cite{au2011short}. 
\end{enumerate}


%To answer the first question, we present the results of a study where we observed the accuracy results of DroidBot on a set from $800$ real app pairs (benign/malicious). For this reproduction study, we take advantage of DroidXP~\cite{DBLP:conf/scam/CostaMCMVBC20}, a tool suite that supported us not only to integrate DroidBot but also in our study setup and data collection. To answer the second question, we hypothesize that the mining sandbox approach for malware detection using tools like DroidBot performs a superficial analysis of differences between app pairs, i.e., only the difference in the set of sensitive APIs called. We assert that there are two major blindspots in the state of the art:
%
%\begin{enumerate}
%    \item The differences between two apps that call the same set of sensitive APIs, but differ in their dynamic call traces between the app's entry point and the sensitive APIs.
%    \item The {\color{blue}change patterns} within the Android Manifest' section that requests permissions, when considering the two variants of apps (i.e., benign and malicious variants).
%\end{enumerate}

%To explore the item (1), we used an auxiliary tool from DroidXP project called DroidXPTrace, which compares all traces from an entry point to sensitive APIs call and check for possible trace divergences. To {\color{blue}mine change patterns} in the Android Manifest files, we harness the standard APK analysis tool from Android~\cite{au2011short}. Altogether, we investigate the following specific questions in our study, for which we provide answers in Section~\ref{sec:results}.
%
%\todo[inline]{\rb{I still have to review the RQs}}
%
%\begin{enumerate}[(RQ1)]
% \item How well does the state of the art in mining sandbox approaches, DroidBot, perform with a large dataset of app pairs with a diverse similarity index?
% \item How relevant is a dynamic call trace analysis to improving malware detection in support of the mine sandbox approach?
% \item How relevant is a manifest files analysis to improving malware detection in support of the mine sandbox approach?
%\end{enumerate}
% 

%First, we reproduced the experiments with the dataset of $102$ apps and achieved a comparable accuracy rate of $63\%$. Our hypothesized blindspots (dynamic call traces and manifest file permission analysis) improve the accuracy rates to $82\%$ and $89\%$ respectively. Secondly, we curated a larger dataset and our findings indicate that the accuracy of DroidBot significantly drops (to $24.12\%$) when reproduced on a larger dataset of $800$ app pairs with a similarity index of $62.47\%$. 

Our experiments show that the accuracy of sandbox approaches does indeed improve, when they are made aware of the differences in the dynamic call trace and difference in permissions requested using the manifest file. Specifically, when considering the dynamic call trace, the accuracy improved by $22\%$ (from $24.12\%$ to $46.12\%$); another $9.63\%$ improvement (from $46.12\%$ to $55.75\%$) could be achieved by taking into account also difference in permissions request. 

An accuracy of $55.75\%$, while being twice as high as the current state of the art, may still appear unsatisfactory for a truly trustworthy sandbox approach. Still, 
our experiments open the discussion (a) on two important blindspots that must be considered when building 
malware detection approaches that mine sandboxes, and (b) on the need to investigate further sources of information when trying to distinguish benign and malign app versions. 
%\kn{Please add here the results of path analysis and manifest files individually and their intersection}
In fact, during our analysis we observed another interesting insight that out of the total of 162 sensitive 
APIs, just 16 ($10.49$\%) were injected at most repackaged apps. 
%\rb{(are all subjects in AndroZoo based on repackaging?)}. 
This indicates that a small set of sensitive APIs is used by malicious developers and hence 
researchers may want to pay more attention to this specific set of sensitive APIs.\\
% to improve the security of Android apps. \\


\noindent
In summary, we make the following contributions:

\begin{enumerate}[1.]
\item A reproduction study of the state of the art in mining sandbox approaches, DroidBot, 
with a larger-scale and more realistic dataset, in terms of number of app pairs and similarity index.
\item A systematic investigation of the effect of trace analysis and static analysis on Android manifest files 
on improving the accuracy of the sandbox approach.
\item An in-depth look into the kind of sensitive APIs that plague most repackaged apps.
\item A reproduction package of our studies is made available online; scripts for statistic analysis are also available.\footnote{https://github.com/droidxp/paper-droidxp-replication.git}
\end{enumerate}


The rest of the paper is organized as follows. We begin with background and related work in Section \ref{sec:background};
readers familiar with dynamic and static analysis on Android apps and mining Android sandbox may skip this section.
Section \ref{sec:experimentalSetup} describes our infrastructure and methodology. Section \ref{sec:results} presents our results and qualitative analysis. Section \ref{sec:threats} discusses thread to validity, and Section \ref{sec:conclusions} 
presents some potential areas for future work and concludes the paper.
%\kn{Here, please describe each section in one sentence}




