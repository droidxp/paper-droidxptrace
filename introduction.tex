\section{Introduction}\label{sec:introduction}

Almost two-thirds of the world use mobile technologies, such as smartphones and tablets, acquired a central role in everyday life at last decade~\cite{Comscore}\cite{DBLP:journals/tse/MartinSJZH17}. In this context, Android Operation System has dominated this market, with around 150 billions of Android application~\footnote{In this paper, we will use the terms Android Applications, Android Apps and Apps interchangeably, to represent Android software applications} (apps) downloaded by October 2018 from its marketplaces (Google Play)~\cite{Statista}. Due to this growing popularity, we witnessed an unprecedented growth of incidents related to Android malicious app (malware). From $2010$ to $2013$, Android malware increase from a hundred samples to more than thousands of samples~\cite{DBLP:journals/comsur/FarukiBLGGCR15,DBLP:journals/csur/SufatrioTCT15}, becoming security issues in Android apps a relevant research topic. Several techniques emerged to identify malicious behaviour and vulnerabilities in Android apps, such as static analysis algorithms to point private information leaks~\cite{DBLP:conf/pldi/ArztRFBBKTOM14}, or reveal misuse of 
cryptographic primitives~\cite{DBLP:journals/tse/KrugerSABM21}

In this context, ... (here we write about sandbox approach)

%maybe this research question will change. We will probably address the false positive issue at this first question.

We investigated the following questions in our study:

\begin{enumerate}[(RQ1)]
 \item What is the false positive frequency at sandbox approach, in terms of benign behaviors unseen at explorer step?
 \item Numerically, which relevant are trace analysis to improve malware detection in support of the mine sandbox approach?
 \item Numerically, which relevant are manifest files analysis to improve malware detection in support of the mine sandbox approach?
\end{enumerate}