\section{Introduction}\label{sec:introduction}

Almost two-thirds of the world use mobile technologies, such as smartphones and tablets, acquired a central role in everyday life at last decade~\cite{Comscore}\cite{DBLP:journals/tse/MartinSJZH17}. In this context, Android Operation System has dominated this market, with around 150 billions of Android application~\footnote{In this paper, we will use the terms Android Applications, Android Apps and Apps interchangeably, to represent Android software applications} (apps) downloaded by October 2018 from its marketplaces (Google Play)~\cite{Statista}. Due to this growing popularity, we witnessed an unprecedented growth of incidents related to Android malicious app (malware). From $2010$ to $2013$, Android malware increase from a hundred samples to more than thousands of samples~\cite{DBLP:journals/comsur/FarukiBLGGCR15,DBLP:journals/csur/SufatrioTCT15}, becoming security issues in Android apps a relevant research topic. Several techniques emerged to identify malicious behaviour and vulnerabilities in Android apps, such as static analysis algorithms to point private information leaks~\cite{DBLP:conf/pldi/ArztRFBBKTOM14}, or reveal misuse of 
cryptographic primitives~\cite{DBLP:journals/tse/KrugerSABM21}

In this context, ...

R1. What is the relationship between similarity coefficient and the accuracy of mine Sandbox approach in terms of detecting malicious behaviors?

R2. How can we use path analysis to leverage malware detection in support of mine sandbox approach?

R3. How complementary is Manifest files analysis with respect to support of mine sandbox approach?