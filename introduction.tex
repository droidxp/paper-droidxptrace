\section{Introduction}\label{sec:introduction}

Mobile technologies like smartphones and tablets have become fundamental to the way we function as a society. Almost two-thirds of the world population
uses mobile technologies~\cite{Comscore,DBLP:journals/tse/MartinSJZH17}, with the
Android Platform dominating this market and accounting for more than 70\% of the \emph{mobile
market share} with almost 3.5 million Android applications~\footnote{In this paper, we will use the terms Android Applications, Android Apps, and Apps interchangeably, to refer to Android software applications} (apps)
available on the Google Play Store~\cite{Statista}. 
With increased popularity, comes increased risk of attacks---motivating efforts from both academia and industry to design and develop new techniques
to identify malicious behavior or vulnerable code in Android apps~\cite{10.1145/3017427}.


One of the most popular classes of malwares are based on repackaging apps~\cite{DBLP:conf/wcre/BaoLL18,le2018towards} where benign
versions of an app from an official app store are
%such as Google Play, 
infected with malicious code, e.g., to broadcast
sensitive information to a private server~\cite{DBLP:journals/tse/LiBK21}, and subsequently shared
with users using different app stores.  The focus of this paper is the Mining Android Sandbox, hereafter \mas, which has been shown effective
in detecting the popular class of Android malware based on 
repackaging benign apps. 
It  takes advantage of automated test case generation tools 
to explore the behavior of an app---in terms of calls to sensitive APIs---and then
generates a sandbox~\cite{DBLP:conf/icse/JamrozikSZ16}. During a normal
execution of the app, the sandbox might block calls to a sensitive API
that had not been observed during the exploratory phase. 


Previous studies~\cite{DBLP:conf/wcre/BaoLL18,DBLP:journals/jss/CostaMMSSBNR22} 
have compared the accuracy of  Android sandboxes for malware detection 
that were produced 
from 
%executing 
different test case generation tools, including Monkey~\cite{Monkey}, DroidBot~\cite{DBLP:conf/icse/LiYGC17}, and Droidmate~\cite{DBLP:conf/kbse/BorgesHZ18} tools.
The studies bring evidence that 
DroidBot outperforms the other test generation tools and leads to sandboxes that are more accurate in detecting malware with a classification rate of 70\%.
But these previous studies have two main limitations.
First, they use a small dataset of malware comprising only \appsSmall pairs of original/repackaged versions of an app. This decision
might compromise the external validity of previous studies. Second, their assessment do not investigate
the impact of repackaged characteristics on the accuracy of the \mas for malware classification, including
(a) whether or not the repackaged version is a malware, (b) the similarity between the original app and the corresponding repackaged malware,
and (c) the malware family (e.g., \fm{gappusin}, \fm{kuguo}, \fm{dowgin}, etc.) when the repackaged
version of an app is a malware.

%{\bf MM: The second part of the last sentence is not clear. Are similarity and malware family two different "malware characteristics"?}

%In this paper, we present the results of an investigation that aims to replicate previous studies~\cite{DBLP:conf/wcre/BaoLL18,DBLP:conf/scam/CostaMCMVBC20} in a larger dataset of app pairs (original/repackaged versions), and then explore whether a more diverse sample of app pairs has an influence on the accuracy of the \mas for malware identification. To this end, we explore the performance of the \mas using a larger dataset we curated for this research and DroidBot~\cite{DBLP:conf/icse/LiYGC17} as test case generation---the tool that, according to the literature, leads to the most accuracy Android sandbox. Our new dataset is an order of magnitude larger (it contains \apps pairs of original/repackaged apps), comprises a much more diverse similarity index, and covers a broader range of malware families. 

To empirically study the impact of these limitations on the reported results, in this paper, we reconsider the performance of the \mas based on
DroidBot~\cite{DBLP:conf/icse/LiYGC17}---as we mentioned, the test case generation tool that, according to the literature, leads to the most accuracy Android sandbox. 
Compared to previous studies~\cite{DBLP:conf/wcre/BaoLL18,DBLP:conf/scam/CostaMCMVBC20},
we use a curated dataset of app pairs (original/repackaged versions) that is, in terms of magnitude, larger than the previously used
dataset (it contains \apps pairs of original/repackaged apps).
 
{\bf Negative results.} Our study reveals a significantly lower
%on a larger and more diverse dataset (compared to previous studies), 
accuracy ($F_1$ score) of the \mas in comparison to what has been reported previously (\fscore versus \fscoreSmall). 
An accuracy of \fscore is clearly unsatisfactory for a trustworthy sandbox approach.
This result motivated us to conduct a series of experiments 
to understand the reasons for the lower accuracy in our larger dataset.
%The original approach classifies an app as malware whenever there exists a difference between the sets of calls to sensitive APIs collected when running the test case generation tool over the original and repackaged versions of the same app.
First, we check the impact of the similarity between original and repackaged versions of
an app on the performance of the \mas for malware classification. Interestingly, our results reveal that there is no
association between similarity and the accuracy of the \mas . 
\kn{Saying there is no association is not the same as saying that similarity does not impact the accuracy. Somehow I feel this statement is very strong. I would rather say that we did not find any significant association between similarity and the accuracy of the \mas.} 
Second, we explore if the
malware familiy could explain the lower performance of the \mas for malware identification in 
our dataset. Our results reveal that the \mas fails to correctly classify most of the samples from
the \gps malware family (a particular adware class frequently appearing in repackaged apps).
Out of the total of \appsGps samples within this family, the \mas failed to correctly classify \appsGpsFN samples as malware (false negative).
Our results reveal that this particular family is responsible for substantially reducing the recall of the \mas in our dataset.
  

%{\bf MM: How is the last sentence different from the one preceding it? It appears to repeat what was just said, IMHO. Also, given that we highlight two distinguishing characteristics of the larger dataset -- similarity of apps and different malware -- I was expecting that we analyse the impact of both the diversity of apps and malware. But we report only that one particular kind of malware is responsible. The dissimilarity of app pairs has no effect? What about other families of malware?}
%
\fh{Still, 
our experiments open the discussion (a) on one important feature found at \gps malware family that should be considered when building malware detection approaches that mine sandboxes, and (b) on the need to have a representative dataset, that should be labeled with sought answers and which should leads close to real-life scenarios.
}


%{\bf MM: I don't get the logical implication that "Still" suggests. Also, (a) and (b) appear out of nowhere... how do they relate to what has been said before in the paragraph? What is blindspot that you are talking about here? How resp. from where do you derive the need to investigate further sources of information?}

The remainder of this paper is organized as follows. 
We provide background information on the \mas and the \mas for malware detection in
Section~\ref{sec:background}. Section~\ref{sec:experimentalSetup}
characterizes our experiments in terms of research goal, questions, metrics, datasets, procedures for collection and data analysis, and environment configuration. Section~\ref{sec:results} and Section~\ref{sec:discussion} present the main findings of our experiments and possible threats to the validity of our results. Finally,
Section~\ref{sec:conclusions} presents concluding remarks and possible future
work. The main artifacts we produced during this research are available in the
paper repository.\footnote{\url{https://anonymous.4open.science/r/paper-droidxptrace-results-F55A/}}
